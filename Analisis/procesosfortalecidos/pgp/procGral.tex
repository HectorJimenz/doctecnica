  \chapter{Proceso Fortalecido de Gestión de Profesores}
\hypertarget{chapter:PFGP}{}
\section{AF-GP Arquitectura}
%\begin{Proceso}{AF-GP}{Arquitectura del proceso fortalecido Gestión de Profesores}{
%		%El Proceso de gestión de profesores corresponde al mecanismo por medio del cual la \refElem{UnidadAcademica}, solicita la información actualizada de los Profesores con el objeto de iniciar la \textbf{Planeación de Horarios}, dicha infromación se solicita a al Dirección de \refElem{CapitalHumano} del Instituto. %		}
%	{PG.X}
%%	\PRccsection{Datos para control interno}
%%	\PRccitem{Versión}{1}
%%	\PRccitem{Elaboró}{Chadwick Carreto Arellano}
%%	\PRccitem{Supervisó}{Jaime López Rabadán}
%%	\PRccitem{Prioridad}{Alta}
%%	\PRccitem{Estatus}{Corrección}
%%	\PRccitem{Complejidad}{Media}
%%	\PRccitem{Volatilidad}{Media}
%%	\PRccitem{Madurez}{Media}
%	\PRsection{Atributos del proceso}
%	\PRitem{Participantes}{
%		\refElem{UnidadAcademica}, \refElem{CapitalHumano}, \refElem{Calmecac}
%	}
%	\PRitem{Objetivo}{
%		Obtener la información actualizada de los profesores para la generación de los horarios con los que se realizara la propuesta de estructura académica.
%	}
%	\PRitem{Interrelación con otros procesos}{	
%		\begin{Titemize}
% 			\Titem Estructura Académica
%		\end{Titemize}
%	}
%	\PRitem{Entradas}{
%		\begin{Titemize}
% 			\Titem Planeación Académica.
% 			\Titem \refElem{Horarios}.
% 			\Titem Horarios Anteriores.
% 			\Titem Profesores de periodos anteriores.
% %		\end{Titemize}
%	}
%	\PRitem{Salidas}{
%		\begin{Titemize}
%           \Titem Propuesta de Horarios con Profesores
% 			\Titem \refElem{Horario}.% 			
%		\end{Titemize}		
%	}
%	
%\end{Proceso}
	La figura \cdtRefImg{pgp:procGral}{AF-GP Arquitectura Fortalecida del Proceso de Gestión de Profesores} muestra el proceso general modificado. Los cambios están centrados en la incorporación del \refElem{Calmecac}, la obtención de la información actualizada por medio de un Servicio Web y en la definición del tipo de profesor que cubrirá las horas que se requieren frente a grupo (interinos, invitados, internos y externos).
	
	Las propuestas que se integrarán en el sistema \refElem{Calmecac} en búsqueda de mejorar el proceso se describen a continuación:  
	
\begin{Citemize}
	\item Se propone que el Calmécac permita cargar la información actualizada de los profesores por medio de un Servicio Web proporcionado por la Dirección de Capital Humano.
	\item Al fortalecer la obtención de información actualizada de los profesores, se podrán generar \hyperlink{PF:PF-PHR}{Horarios} y la  \hyperlink{PF:PF-EA}{Estructura Académica} de una manera más rápida y con esto se conseguirá un mayor nivel de integridad en la información del Calmécac.
	
		La descripción detallada de estas mejoras se encuentra en la sección~\ref{section:PF-GP:validacion}.
	
	\pagebreak
	%-------------------------------------------
	%Diagrama de arquitectura fortalecida
	\Pfig[1]{procesosfortalecidos/pgp/imagenes/procGral}{pgp:procGral}{AF-GP Arquitectura Fortalecida del Proceso de Gestión de Profesores. Para poder leer éste diagrama ir a la sección \ref{section:CodigoColores} }.	
\end{Citemize}
%||||||| .r404
%\begin{Proceso}{PF}{Gestión de Profesores}{
%		%Narrrar el PF-Arquitectura de Procesos
%		
%	}
%	{PG.X}
%	\PRccsection{Datos para control interno}
%	\PRccitem{Versión}{1}
%	\PRccitem{Autor}{Chadwick Carreto Arellano}
%	\PRccitem{Evaluador}{Jose Jaime Lopez Rabadan}
%	\PRccitem{Prioridad}{Alta}
%	\PRccitem{Estatus}{Corrección/Aprobado}
%	\PRccitem{Complejidad}{Alta/Media/Baja}
%	\PRccitem{Volatilidad}{Alta/Media/Baja}
%	\PRccitem{Madurez}{Alta/Media/Baja}
%	\PRccsection{Control de cambios}
%	\PRccitem{Versión 1}{
%			\begin{Titemize}
%				%\RCitem{ PC1}{Corregir la ortografía}{\DONE}
%				%\TODO es para solicitar un cambio \TOCHK Es para informar que se atendió el TODO(ya hizo las correcciones),\DONE Es para indicar que el e valuador reviso los cambios.
%			\end{Titemize}
%	}
%	\PRitem{Participantes}{
%		 Unidad Academica \refElem{UnidadAcademica}, CALMÉCAC\refElem{CALMECAC}, Dirección de Capital Humano\refElem{CapitalHumano}.
%	}
%	\PRitem{Objetivo}{
%		DEfinir el proceso de obtencion de profesores para poder generar en primera instancia la planeación de los Horarios y en segunda instancia para definir la estructura académica.
%	}
%	\PRitem{Interrelación con otros procesos}{	
%		\begin{Titemize}
% 			\Titem Horarios.
% 			\Titem Estructura Académica.
%		\end{Titemize}
%	}
%	\PRitem{Entradas}{
%		\begin{Titemize}
% 			\Titem Nombre 
% 			\Titem RFC
% 			\Titem Número de empleado
% 			\Titem Grado Académico
% 			\Titem Especialidad
% 		    \Titem Fecha de Ingreso
% 			\Titem Carga máxima 
% 			\Titem Turno / Horario Laboral
% 			\Titem Horas Base
% 			\Titem Horas Interinato
% 			\Titem Horas de Actividades Complementarias
% 			\Titem Categoría/Dictamen
% 			\Titem Carga Máxima
% 			\Titem Horas frente a grupo
% 			\Titem Descarga Académica
% 			\Titem Clave de Unidad Académica
% 			\Titem Nombre de la Unidad Académica
% 			\Titem Materias Impartidas
%		\end{Titemize}		
%	}
%	\PRitem{Salidas}{
%		\begin{Titemize}
% 			\Titem 	Información de profesores para Planeación de Horarios.
%		\end{Titemize}		
%	}
%	
%\end{Proceso}
%=======
%
%\chapter{Proceso fortalecido Gestión de Profesores}
%\label{ch:PF-GP}
%
%\begin{Proceso}{PF}{Obtención de profesores}{
%		%Narrrar el PF-Arquitectura de Procesos
%		
%	}
%	{PG.X}
%	\PRccsection{Datos para control interno}
%	\PRccitem{Versión}{1}
%	\PRccitem{Autor}{Chadwick Carreto Arellano}
%	\PRccitem{Evaluador}{Jose Jaime Lopez Rabadan}
%	\PRccitem{Prioridad}{Alta}
%	\PRccitem{Estatus}{Corrección/Aprobado}
%	\PRccitem{Complejidad}{Alta/Media/Baja}
%	\PRccitem{Volatilidad}{Alta/Media/Baja}
%	\PRccitem{Madurez}{Alta/Media/Baja}
%	\PRccsection{Control de cambios}
%	\PRccitem{Versión 1}{
%			\begin{Titemize}
%				%\RCitem{ PC1}{Corregir la ortografía}{\DONE}
%				%\TODO es para solicitar un cambio \TOCHK Es para informar que se atendió el TODO(ya hizo las correcciones),\DONE Es para indicar que el e valuador reviso los cambios.
%			\end{Titemize}
%	}
%	\PRitem{Participantes}{
%		 Unidad Academica \refElem{UnidadAcademica}, CALMÉCAC\refElem{CALMECAC}, Dirección de Capital Humano\refElem{CapitalHumano}.
%	}
%	\PRitem{Objetivo}{
%		DEfinir el proceso de obtencion de profesores para poder generar en primera instancia la planeación de los Horarios y en segunda instancia para definir la estructura académica.
%	}
%	\PRitem{Interrelación con otros procesos}{	
%		\begin{Titemize}
% 			\Titem Horarios.
% 			\Titem Estructura Académica.
%		\end{Titemize}
%	}
%	\PRitem{Entradas}{
%		\begin{Titemize}
% 			\Titem Nombre 
% 			\Titem RFC
% 			\Titem Número de empleado
% 			\Titem Grado Académico
% 			\Titem Especialidad
% 		    \Titem Fecha de Ingreso
% 			\Titem Carga máxima 
% 			\Titem Turno / Horario Laboral
% 			\Titem Horas Base
% 			\Titem Horas Interinato
% 			\Titem Horas de Actividades Complementarias
% 			\Titem Categoría/Dictamen
% 			\Titem Carga Máxima
% 			\Titem Horas frente a grupo
% 			\Titem Descarga Académica
% 			\Titem Clave de Unidad Académica
% 			\Titem Nombre de la Unidad Académica
% 			\Titem Materias Impartidas
%		\end{Titemize}		
%	}
%	\PRitem{Salidas}{
%		\begin{Titemize}
% 			\Titem 	Información de profesores para Planeación de Horarios.
%		\end{Titemize}		
%	}
%	
%\end{Proceso}
%>>>>>>> .r426
%
%	La descripción detallada de esta mejora se encuentra en la sección~\ref{sec:PF-GP:obtenciondeprofesores}.
%
%	\Pfig[1]{procesosfortalecidos/pgp/procGral}{afpgp:procGral}{AF-GP Arquitectura del proceso fortalecido de Gestión de Profesores.Para poder leer este diagrama ir a la seccion \ref{section:CodigoColores}}
%
