\subsection{Valiación de cambios del proceso Estructura Académica}
%Perfil actual: Se espera poder poner el área, donode no se tenga certeza de el área especifica se puede poner de forma general.
%Tipo: básicamente son tres Cambio, Nuevo o Eliminar 


%%------------- Acuerdo 1 -------------
%\hrule
%\vspace{0.2cm}
%\begin{Cdescription}
%	\item[Subproceso:] Registrar propuesta de Estructura Académica
%	\item[Situación actual:] Cada unidad académica se encarga de registrar la propuesta de EA a través del SIIEE de manera manual, esta tarea la desempeñan diferentes áreas o departamentos según cada UA. La Estructura Académica se registra con la finalidad de que la DGyCE la analice y la apruebe.  En el SIIEE solamente se registran profesores de base.
%	\item[Perfil actual:] Unidad Académica - Dirección de Capital Humano
%	\item[Solución propuesta:] Las unidades académicas se encargan de registrar la propuesta de EA en el Calmécac de una manera asistida, donde se capturan tanto profesores de base como interinos provisionales para indicar la necesidad de la UA, diferenciándolos por un estatus, todo esto con el fin de centralizar la información de todos los profesores para no capturar la misma información en varios sistemas.
%	\item[Perfil propuesto:] Unidad Académica - Calmécac
%	\item[Tipo:] Cambio.
%\end{Cdescription}

%%------------- Acuerdo 2 -------------
%\hrule
%\vspace{0.2cm}
%\begin{Cdescription}
%	\item[Subproceso:] Identificar inconsistencias 
%	\item[Situación actual:] Una vez que la Unidad Académica ha registrado la estructura en el SIIEE, la DGyCE  comienza el análisis de la información capturada en el SIIEE y  notifica las correcciones o aclaraciones pertinentes a las unidades, inclusive se llega a requerir que el responsable de estructura de la Unidad Académica acuda a las instalaciones de la DES para revisar a detalle la estructura.
%	A pesar de que se apoyan de reportes y hojas de cálculo, gran parte del análisis se hace a mano.
%	\item[Perfil actual:] Unidad Académica - Dirección de Educación Superior
%	\item[Solución propuesta:] Una vez que la Unidad Académica haya registrado la estructura académica en el Calmécac, éste podrá notificar a la Unidad Académica de las validaciones que no se cumplan en primera instancia, una vez que se cumplan se le notificará a la DES  por medio del Calmécac  que ya puede comenzar con el análisis.
%	\item[Perfil propuesto:] Calmécac
%	\item[Tipo:] Nuevo.
%\end{Cdescription}

%------------- Acuerdo 1 -------------
\hrule
\vspace{0.2cm}
\begin{Cdescription}
	\item[Subproceso:] Registrar soporte documental
	\item[Situación actual:] Cada \refElem {UnidadAcademica} se encarga de enviar a la \refElem{DGYCE} de manera física la documentación necesaria para justificar el por qué un profesor no cubre su carga máxima de acuerdo a su \refElem{DictamenDeCategoria} y en caso de que la Unidad Académica cuente con profesores interinos recurrentes, se envían las cartas compromiso especificando las actividades a las que el profesor se compromete a realizar en el siguiente \refElem {PeriodoEscolar}.
	\item[Perfil actual:] Unidad Académica - Dirección de Educación Superior
	\item[Solución propuesta:] La Unidad Académica se encargará de recopilar la documentación necesaria para justificar su \refElem{EstructuraAcademica} , la deberá digitalizar y adjuntar los documentos en el Calmécac para que la \refElem{DES} proceda con la evaluación de la misma.
	\item[Perfil propuesto:] Unidad Académica - Calmécac
	\item[Tipo:] Nuevo.
\end{Cdescription}

%------------- Acuerdo 2 -------------
\hrule
\vspace{0.2cm}
\begin{Cdescription}
	\item[Subproceso:] Evaluar estructura académica
	\item[Situación actual:] La DGyCE se encarga de evaluar la estructura académica identificando las siguientes inconsistencias por medio de información que proporciona el SIIEE:
	\begin{itemize}
		\item Traslapes en horarios, profesores y espacios.
		\item Que los profesores cubran su carga máxima de acuerdo a su dictamen de categoría y si no la cubren, debe haber un soporte documental.
		\item Que los profesores no rebasen las 40 horas oficiales a la semana.
		\item Materias por grupo sin cubrir.
	\end{itemize}	 
	Las inconsistencias detectadas se le notifican a la Unidad Académica para que las corrija o justifique por qué no proceden. A pesar de los analistas de estructura de la DGyCE que se apoyan de reportes y hojas de cálculo, gran parte del análisis se hace a mano.
	\item[Perfil actual:] Unidad Académica - Dirección de Educación Superior
	\item[Solución propuesta:] Las inconsistencias se detectan desde que la Unidad Académica genera su propuesta de horario en el Calmécac, por lo que la DGyCE sólo se encargará de corroborar que el soporte documental registrado en el Calmécac, justifique las horas adeudadas de un profesor y los compromisos de un profesor interino recurrente.
	\item[Perfil propuesto:] Calmécac - Dirección de Educación Superior.
	\item[Tipo:] Nuevo.
\end{Cdescription}

%------------- Acuerdo 3 -------------
\hrule
\vspace{0.2cm}
\begin{Cdescription}
	\item[Subproceso:] Registrar horas autorizadas
	\item[Situación actual:] La Dirección de Capital Humano se encarga de notificar a la Unidad Académica el número de horas autorizadas por medio de un analista asignado a dicha Unidad, de manera paralela la DCH habilita el acceso al \refElem{SRN} para que la Unidad Académica registre a los profesores interinos de incidencia. Los profesores invitados no se registran de manera oficial.
	\item[Perfil actual:] Dirección de Capital Humano
	\item[Solución propuesta:] La Dirección de Capital Humano se encargará de registrar en el Calmécac las horas autorizadas, teniendo la posibilidad de modificar la cantidad si posteriormente se le autorizan más horas a la Unidad Académica. Esas horas registradas deberán empatar con las horas asignadas a los profesores interinos.
	\item[Perfil propuesto:] Calmécac - Dirección de Capital Humano
	\item[Tipo:] Cambio.
\end{Cdescription}

%------------- Acuerdo 4 -------------
\hrule
\vspace{0.2cm}
\begin{Cdescription}
	\item[Subproceso:] Alimentar al SIIEE
	\item[Situación actual:] Toda la estructura académica qu corresponde a profesores de base se registra en el SIIEE, mientras que los profesores interinos se registran en el SRN. Las unidades académicas se encargan de proveerlo de información para que la DES pueda realizar su análisis, sin embargo la información que contienen estos sistemas no se comparte con ningún otro.
	\item[Perfil actual:] Dirección de Capital Humano
	\item[Solución propuesta:] El Calmécac se encargará de centralizar la información y proporcionar la información de estructura académica al SIIEE, ya que las funtes de información que proveerán al Calmécac son institucionales.
	\item[Perfil propuesto:] Calmécac
	\item[Tipo:] Nuevo.
\end{Cdescription}

%------------- Acuerdo 5 -------------
\hrule
\vspace{0.2cm}
\begin{Cdescription}
	\item[Subproceso:] Registrar profesores interinos
	\item[Situación actual:] La relación de la estructura académica de profesores interinos se registra en el SRN. Actualmente no se tiene un registro oficial de todos los profesores que conforman la estructura académica.
	\item[Perfil actual:] Dirección de Capital Humano
	\item[Solución propuesta:] En el Calmécac se registrarán los profesores interinos con su respectivo tipo de interinato para centralizar la información y  que esté al alcance de las instancias que la requieran.
	\item[Perfil propuesto:] Calmécac
	\item[Tipo:] Nuevo.
\end{Cdescription}
	
	%------------- Acuerdo 6 -------------
	\hrule
	\vspace{0.2cm}
	\begin{Cdescription}
		\item[Subproceso:] Alimentar al SRN
		\item[Situación actual:] Cuando la estructura académica se registra en el SIIEE quedan datos faltantes de profesores, lo que resulta como una necesidad de la Unidad Académica, esos faltantes en la estructura se pasan automaticamente al SRN, por lo que las unidades académicas completan su estructura académica registrando a los profesores interinos en el SRN.
		\item[Perfil actual:] Dirección de Capital Humano
		\item[Solución propuesta:] El Calmécac centralizará la información de estructura académica y proporcionará la información correspondiente al SRN para que la DCH mantenga la información que necesita de los profesores.
		\item[Perfil propuesto:] Calmécac
		\item[Tipo:] Nuevo.
\end{Cdescription}
%------------- Acuerdo 2 -------------
%\hrule
%\vspace{0.2cm}
%\begin{Cdescription}
%	\item[Subproceso:]  Identificar inconsistencias
%	\item[Situación actual:] Las inconsistencias las verifica la DGyCE y consisten en: 
%	1) Revisar que todos los docentes cubran la carga máxima reglamentaria establecida en el RCITPA- IPN, de no ser así debe identificar las causas: La UA debe presentar soporte documental
%	2) Agotar toda posibilidad de que las horas en interinato que solicita la UA hayan sido asignadas a docentes de base con algún adeudo a su carga reglamentaria.
%	Para realizar estas tareas los analistas de estructura se apoyan principalmente de cuatro reportes generados por el SIIEE
%	Estos cuatro formatos no cubren todas las necesidades de la DGyCE para identificar las necesidades, para eso el analista técnico hace una serie de consultas en una BD que él implementó y de esa manera obtienen datos útiles para los analistas de estructura educativa.
%	Con base en esos formatos generados de la consulta, la DGyCE obtiene un listado de docentes que posiblemente pueden cubrir las necesidades que tiene la Unidad Académica, después la depuran manualmente y ese listado lo mandan a cada Unidad Académica como apoyo para la reestructuración de la estructura académica. 
%	\item[Perfil actual:] Unidad Académica - Dirección de Educación Superior
	%\item[Solución propuesta:] Las primeras inconsistencias que se pueden detectar de manera automática es un conteo de horas frente a grupo para verificar que el profesor está en carga máxima de acuerdo al dictamen proporcionado con la información de la DCH. 
	%El Calmécac permitrá adjuntar los soportes documentales para justificar el no tener carga máxima, esto permitirá agilizar tiempos y mantener centralizada la información de cada profesor.
	%\item[Perfil propuesto:] Unidad Académica - Calmécac
	%\item[Tipo:] Cambio.
%\end{Cdescription}



