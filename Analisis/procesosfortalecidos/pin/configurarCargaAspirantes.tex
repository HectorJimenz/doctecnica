% !TeX spellcheck = es_ES
\begin{Proceso}{SPR-1}{Configurar parámetros de Carga}{
			Subproceso en el que el supervisor del \refElem{DepartamentoDeRegistroYSupervisionEscolar} define los parámetros que el \refElem{Calmecac} utiliza para cargar los aspirantes que se recibieron por parte del \refElem{DepartamentoDeInformaticaYEstadisticaEscolar}.
	}
	{SPR.1.X}
	\PRccsection{Datos para control interno}
	\PRccitem{Versión}{1}
	\PRccitem{Autor}{Francisco Isidoro Mera Torres}
	\PRccitem{Evaluador}{Ulises Veléz Saldaña}
	\PRccitem{Prioridad}{Alta}
	\PRccitem{Estatus}{Corrección}
	\PRccitem{Complejidad}{Media}
	\PRccitem{Volatilidad}{Baja}
	\PRccitem{Madurez}{Media}
	\PRccsection{Control de cambios}
	\PRccitem{Versión 0}{
			\begin{Titemize}
				%\RCitem{ PC1}{Corregir la ortografía}{\DONE}
				%\TODO es para solicitar un cambio \TOCHK Es para informar que se atendió el TODO(ya hizo las correcciones),\DONE Es para indicar que el e valuador reviso los cambios.
			\end{Titemize}
	}
	\PRitem{Participantes}{
		 Supervisor del \refElem{DepartamentoDeRegistroYSupervisionEscolar},\refElem{Calmecac}
	}
	\PRitem{Objetivo}{
		Definir los parámetros que el \refElem{Calmecac} utilizará para cargar los aspirantes.
	}
	\PRitem{Interrelación con otros procesos}{	
		\begin{Titemize}
 			\Titem Asignación y Registro de Aspirantes a Unidad Académica
		\end{Titemize}
	}
	\PRitem{Proveedores}{
		 \refElem{DepartamentoDeInformaticaYEstadisticaEscolar}
	}
	\PRitem{Entradas}{
		\begin{Titemize}
 			\Titem Asignación de Aspirantes
		\end{Titemize}		
	}
	\PRitem{Consumidores}{
		\refElem{DepartamentoDeRegistroYSupervisionEscolar}
	}
	\PRitem{Salidas}{
		\begin{Titemize}
 			\Titem Parámetros para realizar la carga de aspirantes.
		\end{Titemize}		
	}
	\PRitem{Precondiciones}{
		\begin{Titemize}
 			\Titem La convocatoria debió haber concluido.
 			\Titem Los aspirantes debieron haber sido asignados a las unidades académicas.
		\end{Titemize}
	}
	\PRitem{Postcondiciones}{
		\begin{Titemize}
 			\Titem El \refElem{Calmecac} tendrá los parámetros que necesita para realizar la carga.
		\end{Titemize}
		
	}
	\PRitem{Frecuencia}{
		Eventual
		% Periódico: Cada cierto tiempo: diario, semanal, anual, etc.
		% Programado: Alguien en algún momento establece la fecha.
		% Eventual: Cada que ocurre un evento que no se puede prever ni programar.
	}
	\PRitem{Tipo}{
		Operación
	}
	% Operacion: Proceso asociado a las actividades propias de la operación del RENIECYT.
	% Mejora continua: Porcesos asociados a actividades de mejora continua del proceso actual.
	% Soporte: Procesos asociados a actividades indirectas necesarias para operar el RENIECYT.
	\PRitem{Áreas de Mejora}{
		\refElem{PI-AO1},\refElem{PI-AO2}
	}

	\PRitem{Factores Críticos}{
			
	}
\end{Proceso}

La figura \cdtRefImg{spr:Configurar}{SPR1-Configurar parámetros de Carga} muestra las actividades que componen el presente subproceso.

\Pfig[1]{procesosfortalecidos/pin/SPR1-Configurar}{spr:Configurar}{SPR1-Configurar parámetros de Carga}


\begin{PDescripcion}
%	\subsubsection{Calmecac}
	%Actor: Calmecac
	\Ppaso \textbf{Calmécac}
	\begin{enumerate}
		\Ppaso Dado que la implementeación del \refElem{Calmecac} aun no esta definida, no podemos realizar aún la descripción de tareas del mismo, pero es importante mencionar que para este proceso, envía al supervisor del \refElem{DepartamentoDeRegistroYSupervisionEscolar} la confuiguración definida previamente y recibe la configuración modificada.
	\end{enumerate}
	
	
	
	%Actor: Departamento de Registro y Supervision Escolar
	\Ppaso \textbf{Departamento de Registro y Supervisión Escolar}
	\begin{enumerate}
		\Ppaso [\Einicio] El subproceso comienza cuando el supervisor del \refElem{DepartamentoDeRegistroYSupervisionEscolar} decide entrar al \refElem{Calmecac} a configurar los parámetros de carga de aspirantes o recibe el mensaje de que la evaluación de la convocatoria ha concluido.
		
		\Ppaso [\itarea]\cdtLabelTask{CUVerificaCargaActual}{CU-Verificar la carga actual:} El supervisor del \refElem{DepartamentoDeRegistroYSupervisionEscolar} desea conocer cuáles son los parámetros que el \refElem{Calmecac} está usando para realizar la carga de aspirantes, si está de acuerdo con estos continúa con la tarea \cdtRefTask{CUAceptarConfiguracionPorDefecto}{CU-Aceptar Configuración por Defecto},si no lo esta continúa con la tarea \cdtRefTask{CUModificarConfiguracionActual}{CU-Modificar Configuración Actual}.
		
		
		\Ppaso [\itarea]\cdtLabelTask{CUAceptarConfiguracionPorDefecto}{CU-Aceptar Configuración por Defecto:} El supervisor del \refElem{DepartamentoDeRegistroYSupervisionEscolar} acepta la configuración que el \refElem{Calmecac} le muestra y termina la operación.
		
		\Ppaso [\itarea]\cdtLabelTask{CUModificarConfiguracionActual}{CU-Modificar Configuración Actual:} El supervisor del \refElem{DepartamentoDeRegistroYSupervisionEscolar} modifica la configuración que el \refElem{Calmecac} le muestra y se lo envía concluyendo está operación.
		
	\end{enumerate}
	
	
\end{PDescripcion}
