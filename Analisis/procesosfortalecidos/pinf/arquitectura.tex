\chapter{Proceso Fortalecido de Gestión de Infraestructura}
\hypertarget{chapter:PFGINF}{}
%========================================================
%Arquitectura de Proceso
%========================================================

%========================================================
% Descripción general del proceso
%-----------------------------------------------
%\begin{Arquitectura}{AG-GINF}{Arquitectura General del Proceso de Gestión de Infraestructura} {
		
		\section{AF-GINF Arquitectura}
		
		
		\noindent La figura \cdtRefImg{arq:PA-RINF}{Arquitectura del Proceso de Gestión de Infrestructura} muestra el proceso general modificado. Los cambios están centrados en la incorporación del \refElem{Calmecac} y su interacción con las Unidades Académicas \footnote{Ver \refElem{UnidadAcademica}} para el registro de la infraestructura y así poder seleccionar los salones a los horarios de las Unidades de Aprendizaje \footnote{Ver \refElem{UnidadDeAprendizaje}}. 
		
		%El proceso de Gestión de Infraestructura se encarga de gestionar los diversos espacios del instituto para la impartición de las clases de las unidades de aprendizaje\footnote{ver \refElem{UnidadDeAprendizaje}}. Al mismo tiempo, se encarga de gestionar los espacios externos, que pertenecen a otras entidades, en las cuales el instituto tiene permitido impartir clases. 
		
		Con la implementación de este proceso, el Instituto podrá contar con un catálogo de todos los espacios en donde se imparten clases, el cual será actualizado por cada Unidad Académica.
		
		La descripción detallada de estas mejoras se encuentra en la sección~\ref{section:PF-GINF:validacion}.
		 
		 \pagebreak
		
		\Pfig[.4]{procesosfortalecidos/pinf/images/arquitectura-infraestructura.png}{arq:PA-RINF}{AF-GINF Arquitectura Fortalecida del Proceso de Gestión de Infrestructura}
