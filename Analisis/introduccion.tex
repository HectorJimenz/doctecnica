\section{Introducción}
Cada día el uso de dispositivos móviles ha ido en aumento, y con ello el uso de aplicaciones móviles también. En 2015 hubo un aumento de 58\% en el uso de aplicaciones móviles [1], ya que actualmente los smartphones son el dispositivo favorito de 9 de cada 10 consumidores. De la misma manera en 2017 se han detectado 10 tendencias en aplicaciones móviles las cuales son: Pagos electrónicos, realidad aumentada, el aumento de Android, el uso de Bitcoins, animaciones dinámicas, internet de las cosas, aplicaciones On-Demand, inteligencia artificial, big data y aplicaciones en la nube. [2]
Así mismo mejorar los servicios que se ofrecen en el ámbito público, de modo que las tecnologías de información sirven para que la mayoría de las personas tengan acceso a este tipo de servicios y se cuente con una mejor calidad en los servicios facilitando el día a día de las personas.
En México existe una problemática y es que se encuentra dentro de una zona geográfica conocida como: Cinturón de Fuego. Esta zona hace que a lo largo del año se tenga actividad sísmica dentro del país y, aunque la mayoría de las veces estas actividades no sean alarmantes, se cuenta con un protocolo por parte de Protección Civil que debe seguirse en caso de que haya un sismo de magnitud alta. Por otro lado, México se encuentra en una zona tropical lo que provoca que en el país también lleguen huracanes y tormentas tropicales y, aunque estas sean la mayoría de las veces en zonas costeras, también debe seguirse un protocolo de Protección Civil. Sin embargo, en estos protocolos no se hace mención sobre donde pueden encontrarse refugios o albergues.
La actividad sísmica en México aumentó en cada enero de los últimos cinco años, al pasar de 490 temblores en 2014 a dos mil 575 en ese mismo tiempo de 2018, revelan datos del Servicio Sismológico Nacional (SSN).
Desafortunadamente, los desastres naturales específicamente en terremotos o sismos ya que estos siempre estarán presentes y generarán una serie de daños y consecuencias para la humanidad en el futuro. Hoy día, por ejemplo, el calentamiento global está provocando cambios climáticos y graves desastres naturales, que tienen un impacto no solo sobre la economía de las naciones y de las personas que las habitan, sino también sobre los sistemas de convivencia y de interrelación social. 
De hecho, es de esperarse que los terremotos o sismos del futuro sean de mayor magnitud e intensidad, provocando daños mayores a la economía de las naciones y de las personas, lo que hace necesario estudiar y analizar desde diferentes disciplinas estos fenómenos para mitigar sus consecuencias negativas 

Desastres Naturales 

La palabra desastre proviene del latín dis (separación) y de astro (estrella), haciendo referencia a fenómenos astrológicos anormales, que los antiguos romanos tomaban como presagio del avecinamiento de grandes males [2]. En este sentido, un desastre natural es un fenómeno anormal de la naturaleza que genera ciertos perjuicios y pérdidas para los seres humanos.
Los desastres naturales pueden clasificarse en cuatro diferentes tipos, de acuerdo con la naturaleza del desastre y la causa que los genera. Estos son los desastres hidrológicos, los meteorológicos, los geofísicos y los biológicos.
Sin embargo, en lo que se centra este trabajo terminal son en desastres geofísicos provienen de la tierra o el espacio, tales como las tormentas solares, los terremotos, las avalanchas, los derrumbes, terremotos, erupciones volcánicas y hundimientos de tierra, entre otros.

\section{Problemática}
El 80\% de las personas que habitan en la Ciudad de México desconocen los protocolos propuestos por protección civil y en muchos casos no se tiene acceso a estos ya sea por falta de tiempo o simplemente por falta de interés. Esto conlleva que a falta de esta información no sepan cómo actuar, a donde dirigirse, donde y como resguardarse cuando suceden estos tipos de desastres naturales y que es lo que se recomienda según protección civil.
\\\\Al igual desconocen información como la ubicación de los centros de recolección de víveres, refugios y albergues. Por lo tanto, la mayoría de las personas no sebe a donde dirigirse a brindar su ayuda y víveres para los afectados después de un sismo. Con esto tenemos que también no se conoce un inventario de los víveres y herramientas que se tienen en estos centros, este ha llegado a ser un problema bastante grande en los últimos sismos de magnitud grande en la Ciudad de México ya que los ciudadanos al desconocer esta información no llevan la ayuda que más se necesita e incluso no la llevan en donde realmente se necesita más.
\\\\Otro de los problemas que se presentan cuando ocurre un sismo es que no tenemos una conexión ya que se cae por completo la red por lo tanto esto provoca que cuando se quiera consultar información relacionado con desastre natural que se estuviese presentando en ese momento.
\\\\Es por eso que en este Trabajo Terminal se propone el desarrollo de un prototipo de aplicación móvil que esté basada en arquitectura offline para que pueda ser utilizada en caso de que se presente un desastre natural anteriormente mencionados y poder resolver la problemática de qué hacer cuando se presentan y a donde ir cuando se presentan.

\section{Objetivo}

\subsection{Objetivo General}
Desarrollar un sistema que proporcione a los usuarios una herramienta de alerta sísmica y de apoyo para visualizar la información de refugios, centros de recolección y albergues dados de alta en apoyo a los sismos en la Ciudad de México, mostrando una ruta a sus destinos mediante dispositivos móviles sin necesidad de una conexión a internet.

\subsection{Objetivos Específicos}
\begin{itemize}
	\item Generación de alertas sísmicas en la Ciudad de México, las cuales serán recibidas mediante frecuencias de radio.
	\item Permitir visualizar los protocolos que se deben seguir durante y después de un sismo, según protección civil.
	\item Permitir obtener la ruta hacia los refugios, centros de recolección y albergues mediante el GPS, sin necesidad de una conexión a internet.
	\item Permitir visualizar la última información de los refugios, centros de recolección y albergues tales como: ubicación, capacidad, disponibilidad de víveres y herramientas.
	\item Desarrollar un módulo de contribución a alta de refugios, centros de recolección y albergues nuevos por los usuarios, la cual deberá ser validada para su posterior publicación.
\end{itemize}


\section{Alcance}

\section{Justificación}
La necesidad que se pretende satisfacer en los usuarios de esta aplicación es la de seguridad, una definición dentro de las ciencias de la seguridad es “Ciencia interdisciplinaria que está encargada de evaluar, estudiar y gestionar los riesgos que se encuentra sometido una persona, un bien o el ambiente”. Con esto tenemos que la aplicación deberá ser capaz de alertar a los usuarios en caso de un sismo en la Ciudad de México, dejando al sistema que se encargue de informar acerca de los protocolos a seguir durante y después de un sismo, según Protección Civil. Al igual de poder brindar información de los refugios, centros de recolección y albergues, como lo es: ubicación, inventarios de víveres, inventario de herramientas, capacidad y nivel de necesidad de ayuda comunitaria, etc. Además de esto sabemos que las telecomunicaciones pueden llegar a caer durante un desastre natural como lo es un sismo, por lo que la necesidad de poder llegar a un albergue, centro de recolección o refugio es muy importante para poder salvaguardarse o brindar la ayuda a quien lo necesita, por esto es necesario tener a la mano una ruta hacia estos lugares sin la necesidad de una conexión a internet por lo que el mapeo offline será un gran punto a integrar en la aplicación. Todo esto para brindarle al usuario todas las herramientas para sobrellevar un ismo en la Ciudad de México. 

\section{Definición de términos}