{\bf Los requerimientos funcionales utilizan una clave RFX}, donde:
	
\begin{description}
	\item[X] Es un número consecutivo: 1, 2, 3, ...
	\item[RF] Es la clave para todos los {\bf R}equerimientos {\bf F}uncionales.\\
\end{description}



	{\bf Los requerimientos del usuario utilizan una clave RUX}, donde:
	
\begin{description}
	\item[X] Es un número consecutivo: 1, 2, 3, ...
	\item[RU] Es la clave para todos los {\bf R}equerimientos del {\bf U}suario.\\
\end{description}



{\bf Los Requerimientos No Funcionales utilizan una clave RNFX},donde:
\begin{description}
	\item[X] Es un número consecutivo: 1, 2, 3, ...
	\item[RNF] Es la clave para todos los {\bf R}equerimientos {\bf N}o {\bf F}uncionales.\\
\end{description}



{\bf Las Reglas de Negocio utilizan una clave RPCX}, donde:
\begin{description}
\item[X] Es un número consecutivo:1,2,3,....
\item[RPC] Es la clave para todos los {\bf R}eglas {\bf P}rotección {\bf C}ivil.\\
\end{description}



{\bf Los Casos de Uso  utilizan una clave CUX},donde:

\begin{description}
	\item[X] Es un número consecutivo: 1, 2, 3, ...
	\item[CU] Es la clave para todos los {\bf C}asos de {\bf U}so.\\
\end{description}


DISEÑO DE INTERFACES:\\
{\bf Las Interfaces de Usuario utilizan una clave IUX},donde:

\begin{description}
	\item[X] Es un número consecutivo: 1, 2, 3, ...
	\item[IU] Es la clave para todas las {\bf I}nterfaces  de {\bf U}suario.\\
\end{description}

{\bf Los Mensages  utilizan una clave MSGX},donde:

\begin{description}
	\item[X] Es un número consecutivo: 1, 2, 3, ...
	\item[MSG] Es la clave para todos los {\bf M}ensages.\\
\end{description}

	Además, para los requerimeitnos funcionales se usan las abreviaciones que se muestran en la tabla~\ref{tbl:leyendaRF}.
\begin{table}[hbtp!]
	\begin{center}
    \begin{tabular}{|r l|}
	    \hline
    	{\footnotesize Id} & {\footnotesize\em Identificador del requerimiento.}\\
    	{\footnotesize Pri.} & {\footnotesize\em Prioridad}\\
    	{\footnotesize Ref.} & {\footnotesize\em Referencia a los Requerimientos de usuario.}\\
    	{\footnotesize MA} & {\footnotesize\em Prioridad Muy Alta.}\\
    	{\footnotesize A} & {\footnotesize\em Prioridad Alta.}\\
    	{\footnotesize M} & {\footnotesize\em Prioridad Media.}\\
    	{\footnotesize B} & {\footnotesize\em Prioridad Baja.}\\
    	{\footnotesize MB} & {\footnotesize\em Prioridad Muy Baja.}\\
		\hline
    \end{tabular} 
    \caption{Leyenda para los requerimientos funcionales.}
    \label{tbl:leyendaRF}
	\end{center}
\end{table}
