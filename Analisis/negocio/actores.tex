%=============================================
% Descripción de actores
\label{chapter:ActoresDelSistema}

	En el presente capítulo se definen los participantes en los procesos actuales y fortalecidos. Se da una breve descripción de cada uno, la mayoría tomada de Protección Civil de la Ciudad de México.
	%Reglamento General, Reglamento interno y Manuales de operación del IPN, entre otros. %Aquellas descripciones tomadas de la normatividad tienen la referencia del documento de dónde fueron tomadas.

\section{Descripción de participantes}

Cada participante se describe con la siguiente estructura:

\begin{objetivos}[Descripción de participantes]
	\item {\bf Nombre:} Ciudadano.
	\item {\bf Descripción:} Cualquier persona que habite en la Ciudad de México, toda persona que desee obtener informacion acerca de los albergues, centros de recoleccion o refugios registrados en la CDMX, que necesite información referente a protocolos de protección civil y que desee recibir alertas en caso de un sismo.
	
	\item {\bf Nombre:} Administrador.
	\item {\bf Descripción:} Persona seleccionada para la gestión de albergues, centros de recolección y refugios en la Ciudad de México al igual de la autorización de las propuestas enviadas por los Ciudadanos. Esta persona debe ser reconocida por Protección Civil de la Ciudad de México.
	
	\item {\bf Nombre:} Encargado.
	\item {\bf Descripción:} Persona seleccionada para la supervision de actividades en un albergue, centro de recolección o refugio en la Ciudad de México. La cual podra actualizar los inventarios, necesidades e información relevante acerca de los mismos.
\end{objetivos}

\section{Participantes detectados}

\begin{actor}{ciudadano}{Ciudadano}{
		Persona que habita en la Ciudad de México.
}	
\end{actor}

\begin{actor}{administrador}{Administrador}{
		Persona encargada de la alta de albergues, centros de recolección y refugios en la Ciudad de México, asi como de la gestión de los mismos.
}	
\end{actor}

\begin{actor}{encargado}{Encargado}{
		Persona que supervisa las actividades de un albergue, centro de recolección o refugio.
}	
\end{actor}

