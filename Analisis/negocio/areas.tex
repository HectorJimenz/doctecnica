	Como resultado de revisar los procesos asociados al SAES, se presentan las áreas de oportunidad y la estrategia que puede implementarse en el Calmécac.
	
	\begin{itemize}
	    \item Se bosquejan los problemas encontrados en los procesos.
	    \item Se presenta la propuesta de alcance a cubrir por parte del Calmécac, la cual deberá ser revisada y aprobada por las áreas involucradas en los diferentes procesos afectados y por el comité que supervisa el proyecto.
	\end{itemize}

	En este capítulo señalamos las áreas de oportunidad, y en los siguientes capítulos describimos la forma en que dichas mejoras son implementadas en cada uno de los procesos fortalecidos.

%------------------------------------------------
\section{Proceso de Gestión de Programas Académicos}
%- - - - - - - - - - - - - - - - - - - - - - - - 

\begin{AreaOportunidad}{PGPA-AO1}
	\item[Área:] {Unidades Académicas\footnote{Ver \refElem{UnidadAcademica} }, \refElem{DES} y \refElem{DAE} }.
	\item[Procesos:] Gestión de Programas Académicos.
	\item[Problema:] La comunicación entre las Unidades Académicas y la \refElem{DES} para compartir la información del registro de nuevos programas académicos o el rediseño de los mismos, se realiza actualmente sin una interacción con el \refElem{SAES}, que es donde al final se registran los planes de estudio oficiales dentro del instituto. Lo anterior deriva en un retardo en el registro de la información oficial y en la detección de errores en la misma.
	\item[Área de oportunidad:] Mejorar la comunicación a través de una interfaz para la gestión de planes de estudio por parte de las Unidades Académicas, donde la \refElem{DES} pueda observar la información y dar su visto bueno, una vez realizada su validación de los planes de estudio por medio del proceso vigente.
	\item[Impacto:] Medio.
	\item[Alcance:] Dentro del ámbito del CALMÉCAC recae considerar que la Unidad Académica pueda gestionar sus planes de estudio o el rediseño de los mismos, el cual pueda ser observado por la \refElem{DES} y avalarlo una vez que ha realizado su proceso de validación. La \refElem{DAE} deberá mantener las facultades de realizar los cambios que sean necesarios para actualizar la información y validar el registro o los cambios realizados.\\
	
\end{AreaOportunidad}

%==================================




%------------------------------------------------
\section{Proceso de Registro de Infraestructura}
%- - - - - - - - - - - - - - - - - - - - - - - - 
\begin{AreaOportunidad}{PGINF-AO1}
	\item[Área:] \refElem{DES}.
	\item[Procesos:] Proceso de Registro de Infraestructura.
	\item[Problema:] 	No se cuenta con la información precisa de donde se imparten las clases debido a que  no se lleva un control adecuado de los espacios, ya sea internos o externos, que están habilitados para impartir clases de los diversos programas académicos que ofrece el Instituto.
	\item[Área de oportunidad:] Implementar un proceso que permita la construcción de un catálogo institucional a partir de la información que cada unidad académica proporcione, en donde se tengan registrados todos los espacios habilitados para impartir clases.  Además, permitiría identificar los espacios que no pertenecer al Instituto, como lo son clínicas u hospitales.  
	
	\item[Impacto:] bajo.
	\item[Alcance:] Implementar un proceso que permita la construcción de un catálogo institucional de los espacios usados para impartir las unidades de aprendizaje, a partir de la información que cada unidad académica proporcione. Un catálogo con estás características permite identificar como se usan desde la perspectiva académica los espacio que pertenecen al Instituto, adicionalmente se podrá identificar aquellos que no le pertenecen  como son clínicas y hospitales.

\end{AreaOportunidad}






%\begin{AreaOportunidad}{PI-AO1}
%	\item[Área:] \refElem{DAE}
%	\item[Procesos:] Admisión e Ingreso.
%	\item[Área de Oportunidad:] El mecanismo actual en el que el Departamento de Informática comunica al Departamento de Supervisión los aspirantes aceptados y su asignación a las Escuelas ({\em la cual se realiza mediante vistas de Oracle que deben ser consultadas manualmente y mediante ``scripts programados''.}) provoca que el trabajo para mantener al SAES actualizado implique mucho trabajo y tiempo.
%	\item[Mejoras:] Mejorar dicha comunicación mediante la integración de ambos sistemas ({\em el de informática para el proceso de admisión y el Calmécac}) mediante Servicios Web, Formatos especializados que permitan automatizar el intercambio de información o la integración de ambas Bases de datos.
%	\item[Impacto:] Medio.
%	\item[Alcance:] Dentro del ámbito del proyecto Calmécac recae considerar la mejora de la comunicación entre estas áreas, pero no, adecuaciones o mejoras al Sistema de Admisión, las cuales de ser necesarias requerirán del apoyo del departamento de Informática para lograrlas.
%\end{AreaOportunidad}




%------------------------------------------------
\section{Proceso de Gestión de Profesores}
%- - - - - - - - - - - - - - - - - - - - - - - - 
\begin{AreaOportunidad}{PGP-AO1}
	\item[Área:] \refElem{DCH}.
	\item[Procesos:] Gestión de Profesores.
	\item[Problema:] Actualmente existen dos sistemas oficiales (SAES y SIEE) que manejan la información de la plantilla docente, dicha información no coincide debido a que se manejan fuentes de información diferentes.
	La información de la situación laboral de los docentes es dinámica, la cual es gestionada por la \refElem{DCH}, el SAES no cuenta con un mecanismo que permita tener dicha información vigente.\\
	Cuando se generan cambios en la estructura académica una vez iniciado el semestre, no existe un mecanismo de notificación y actualización de la información para \refElem{DAE}, \refElem{DES} y \refElem{DEMS}.
	
	%Además el concepto de Profesor Invitado no se contempla en el sistema SAES.
	%\item[Área de oportunidad:] 
	\item[Impacto:] Medio.
	\item[Alcance:] Se propone obtener la información de los profesores directamente de la \refElem{DCH}. %Una vez que la estructura académica ha sido aprobada por Capital Humano, la \refElem{DES}  y la \refElem{DEMS}, se generará un mecanismo que informe de los cambios realizados a éstas por parte de la unidad académica, con la finalidad de enterar a la DES y Capital Humano
	
\end{AreaOportunidad}


%=============================



%------------------------------------------------
\section{Proceso de Registro de Horarios}

\begin{AreaOportunidad}{PRH-AO1}
	\item[Área:] \refElem{DES}, \refElem{JefeDeDepartamento}, \refElem{SubdireccionAcademica}.
	\item[Procesos:] Registro de Horarios.
	
	\item[Problema:] La generación de horarios por parte de las unidades académicas se realiza en sistemas independientes al SAES, posteriormente se capturan en este, sin embargo el SAES no brinda apoyo en identificar traslapes entre salones, profesores, que una unidad de aprendizaje que pertenece a un grupo no se traslape en el mismo horario con otra unidad de aprendizaje del mismo grupo. 
	
	\item[Área de Oportunidad:]Desarrollar un Subsistema que pueda enriquecer la estructura académica. Con esto se desacoplaría la parte de profesores y carga académica con fines presupuestales y laborales, permitiendo tener disponible la información que requiere el \refElem{Calmecac} para la Gestión Escolar.
		\begin{itemize}
			\item Identificar los profesores con traslapes (entre, carreras)
			\item Las aulas que se traslapan.
			\item Identificar los profesores que no tienen su carga máxima e acuerdo a su categoría.
			\item Conocer el catálogo de materias impartidas por un profesor.
			\item Contar en el sistema con el soporte documental que justifica la carga (cumplida o no) de un profesor (cuando no es frente a grupo).
			\item Que el sistema sugiera que materias puede impartir un profesor si no cubre su carga máxima considerando:
				\begin{enumerate}
					\item Materias impartidas.
					\item Horario del profesor.
					\item Materias disponibles.
				\end{enumerate}
			\item Que el sistema considere que hay más de una clave presupuestal y tipo de contrato por profesor.
			\item El Calmécac proveerá parte de la funcionalidad del SIIEE y una interfaz de comunicación con el SIEE para evitar la doble captura, así el SIEE podrá consultar o adquirir la última versión de la Estr. Acad. cuando lo desee.
		\end{itemize}
	
	
	%No existe el concepto de carga de cada profesor, mínima ni máxima. estructura academica
	%- El SAES no permite verificar, controlar, identificar, reportar, profesores que no están cubriendo la carga que les corresponde ni la causal de dicha situación que permita al personal de la Unidad  o CH actuar de manera correcta.
	
	
	
	%\item[Área de oportunidad:]
	%	Generación asistida en la generación de  plantillas grupales que incluyan unidades de Aprendizaje, horarios preestablecidos indicando: días, horas y espacios para la impartición de las diferentes unidades de aprendizaje, permitiendo la selección y asignación del profesor a cubrir esa unidad de aprendizaje.\\
	%Consulta automática de información actualizada de los profesores a través de un servicio web.\\
	%\\
	
	%Evitar traslapes en asignación de horarios por profesor.\\
	%\\
	%.\\
	
	\item[Impacto:] Alta.
	
	\item[Alcance:] Dentro del ámbito del proyecto CALMÉCAC recae considerar las siguientes acciones:
	\begin{itemize}
		\item Copiar una estructura de un semestre similar o equivalente  para usarlos como propuesta o plantilla de horarios.
		\item Evitar traslapes en asignación de horarios por grupo.
		\item Evitar traslapes en asignación de espacios e infraestructura académica.
		\item Ofrecer un mecanismo que permita modificar un horario establecido de acuerdo a las necesidades de la unidad académica.
	\end{itemize}
	
\end{AreaOportunidad}
%=======================



%------------------------------------------------
\section{Proceso de Gestión de Estructura Académica}
%- - - - - - - - - - - - - - - - - - - - - - - - 
\begin{AreaOportunidad}{PEA-AO1}
	\item[Área:] \refElem{UnidadAcademica}.
	\item[Procesos:] Solicitud de horas de interinato.
	\item[Problema:] Las escuelas solicitan horas de interinato sin agotar las capacidades de la planta docente.
	\item[Área de Oportunidad:] Generar un mecanismo que ayude a identificar a los profesores que no cubren su carga máxima.
		\item[Impacto:] Medio
	\item[Alcance:] Implementación de una estrategia que facilite identificar a los profesores que no cubren su carga maxima.
	\end{AreaOportunidad}


%- - - - - - - - - - - - - - - - - - - - - - - -

\begin{AreaOportunidad}{PEA-AO2}
	\item[Área:] \refElem{UnidadAcademica},\refElem{DAE},\refElem{DCH}
	\item[Procesos:] Registro de estructura académica, Registro de horarios en el SAES, Registro en el SRN
	\item[Problema:] Toda la información que contiene la estructura se captura más de una ocasión manualmente y se hace para cada sistema (SIIEE, SRN y SAES), por lo que es no es complicado contar con una versión única de la información de los profesores, situación que empeora cuando en alguno de os sistemas se realiza un cambio, éste no se refleja en los demás sistemas.	
		\item[Área de Oportunidad:]Desarrollar un mecanismo que permita sincronizar los diferentes sistemas que se utilizan en el Instituto que están relacionados con la gestión de profesores y su asignación de unidades académicas.
	\item[Impacto:] Medio
	\item[Alcance:] Generación de estrategias de comunicación, como pueden ser los servicios web.
\end{AreaOportunidad}

%- - - - - - - - - - - - - - - - - - - - - - - - -
%
\begin{AreaOportunidad}{PEA-AO3}
	\item[Área:] \refElem{UnidadAcademica}.
	\item[Procesos:] Identificación de inconsistencias
	\item[Problema:] No contar con el soporte documental cuando la carga máxima de un profesor esta por arriba o por debajo de la carga máxima.
	\item[Área de Oportunidad:]Generar un repositorio que facilite el acceso al soporte documental de los casos de los profesores tienen una carga diferente a la m-axima.
	\item[Impacto:] Medio
	\item[Alcance:] Generar un mecanismo de consulta que permita conocer porque un profesor no cubre su carga máxima.
\end{AreaOportunidad}
%
%%- - - - - - - - - - - - - - - - - - - - - - - - -
%
%
%\begin{AreaOportunidad}{PEA-AO4}
%	\item[Área:] \refElem{UnidadAcademica},\refElem{DES}.
%	\item[Procesos:] Generación de propuesta de estructura académica.
%	\item[Problema:] Que un profesor tenga adeudo de horas, es decir, que no esté en carga máxima y no justifique el por qué.
%	\item[Impacto:] Que no se permita si no tiene un soporte documental ya que se puede prestar a favoritismos.
%	\item[Alcance:] Agregar las validaciones necesarias para la asignación de unidades de aprendizaje a profesores.
%\end{AreaOportunidad}


%- - - - - - - - - - - - - - - - - - - - - -  - - - -  - -

\begin{AreaOportunidad}{PEA-AO4}
	\item[Área:] \refElem{}.
	\item[Procesos:] Evaluación de la estructura académica, Envío de propuesta de reestructuración académica
	\item[Problema:] Gestionar la aprobación de la estructura académica es complejo por la cantidad de variable que están inmersas en el proceso, el mecanismo actual de comunicación es deficiente.
	\item[Área de Oportunidad:]Construir una estrategia que ayude a facilitar la gestión de la estructura Académica.
	\item[Impacto:] Alto
	\item[Alcance:] Construir un módulo que ayude a la gestión de la estructura académica.
\end{AreaOportunidad}


%- - - - - - - - - - - - - - - - - - - - - - - - - - 





%------------------------------------------------
\section{Procesos de Ingreso}
%- - - - - - - - - - - - - - - - - - - - - - - - - -

\begin{AreaOportunidad}{PI-AO1}
	\item[Área:] \refElem{DAE}
	\item[Procesos:] Admisión e Ingreso.
	\item[Área de Oportunidad:] El mecanismo actual en el que el Departamento de Informática comunica al Departamento de Supervisión los aspirantes aceptados y su asignación a las Escuelas ({\em la cual se realiza mediante vistas de Oracle que deben ser consultadas manualmente y mediante ``scripts programados''.}) provoca que el trabajo para mantener al SAES actualizado implique mucho trabajo y tiempo.
	\item[Mejoras:] Mejorar dicha comunicación mediante la integración de ambos sistemas ({\em el de informática para el proceso de admisión y el Calmécac}) mediante Servicios Web, Formatos especializados que permitan automatizar el intercambio de información o la integración de ambas Bases de datos.
	\item[Impacto:] Medio.
	\item[Alcance:] Dentro del ámbito del proyecto Calmécac recae considerar la mejora de la comunicación entre estas áreas, pero no, adecuaciones o mejoras al Sistema de Admisión, las cuales de ser necesarias requerirán del apoyo del departamento de Informática para lograrlas.
\end{AreaOportunidad}

%- - - - - - - - - - - - - - - - - - - - - - - -

\begin{AreaOportunidad}{PI-AO2}
	\item[Área:] \refElem{DAE}.
	\item[Procesos:] Admisión e Ingreso.
	\item[Área de Oportunidad:] El mecanismo con el que el Departamento de Informática envía al Departamento de Supervisión las actualizaciones de boleta derivadas de la revisión de expedientes ({\em la cual se realiza mediante vistas de Oracle que deben ser consultadas manualmente y mediante ``scripts programados''}) ha provocado que se entreguen boletas asignadas a alumnos cuyo numero de boleta no ha sido cargado al SAES. Esto es importante debido a que el número de boleta es utilizado en el SAES para garantizar el acceso a todos los servicios que otorga el IPN.
	\item[Mejoras:] Mejorar dicha comunicación mediante la integración de ambos sistemas ({\em el de informática para el proceso de admisión y el Calmécac}) mediante Servicios Web, Formatos especializados que permitan automatizar el intercambio de información o la integración de ambas Bases de datos en menor tiempo, con menor esfuerzo y reducir la intervención del personal.
	\item[Impacto:] Medio.
	\item[Alcance:] Dentro del ámbito del proyecto Calmécac recae considerar la mejora de la comunicación entre estas áreas, pero no, adecuaciones o mejoras al Sistema de Admisión, las cuales de ser necesarias requerirán del apoyo del departamento de Informática para lograrlas.
\end{AreaOportunidad}
%- - - - - - - - - - - - - - - - - - - - - - - - 

\begin{AreaOportunidad}{PI-AO3}
	\item[Área:] \refElem{DAE}.
	\item[Procesos:] Ingreso e inscripción.
	\item[Área de oportunidad:] El mecanismo con el que las Escuelas inscriben en el SAES a los alumnos de nuevo ingreso es manual, uno por uno, llenando grupos bajo criterios que son ``usos y costumbres'' o ``políticas ampliamente utilizadas''. Algunos de ellos son: por genero, localidad, examen de conocimientos, si el alumno trabaja, etc. El problema es que no hay un criterio homogeneizado y el trabajo y tiempo que implica esta actividad.
	\item[Propuesta:] Desarrollar en el Calmécac un módulo que facilite esta tarea, para los criterios más utilizados o comunes, con el fin de homogeneizar criterios, reducir el error humano y reducir el esfuerzo invertido en esta tarea.
	\item[Impacto:] Medio.
	\item[Alcance:] Dentro del ámbito del proyecto Calmécac.
\end{AreaOportunidad}

%- - - - - - - - - - - - - - - - - - - - - - - - 
\begin{AreaOportunidad}{PI-AO4}
	\item[Área:] \refElem{DAE}.
	\item[Procesos:] Admisión e Ingreso.
	\item[Área de oportunidad:] Actualmente cuando un alumno ha infringido algún lineamiento que lo deja fuera del instituto de manera permanente, el SAES no cuenta con un mecanismo para consultar dicha situación o para prevenir un reingreso de dicho alumno.
	\item[Propuesta:] Que el Calmécac cuente con el registro de alumnos en esta situación para conocer el estado final de cada alumno, así como prevenir su reingreso.
	\item[Impacto:] Medio.
	\item[Alcance:] Dentro del ámbito del proyecto Calmécac.
\end{AreaOportunidad}

%- - - - - - - - - - - - - - - - - - - - - - - - 

\begin{AreaOportunidad}{PI-AO5}
	\item[Área:] \refElem{DAE}.
	\item[Procesos:] Admisión, Ingreso e Inscripción.
	\item[Área de oportunidad:] La estructura actual del SAES dificulta la operación cuando un alumno pertenece a más de un programa académico, plan de estudios o Unidad Académica. 
	\item[Propuesta:] En el Calmécac considerar que un alumno puede estar participando en más de un Programa académico, en mas de una modalidad, en más de un plan de estudios y en mas de una unidad académica.
	\item[Impacto:] Medio.
	\item[Alcance:] Dentro del ámbito del proyecto Calmécac.
\end{AreaOportunidad}

%- - - - - - - - - - - - - - - - - - - - - - - - 

\begin{AreaOportunidad}{PI-AO6}
	\item[Área:] \refElem{DAE}.
	\item[Procesos:] Asignación de boleta.
	\item[Área de oportunidad:] Los cambios implementados en la estructura y manejo de Número de Boleta dificulta su uso en las Unidades Académicas para la generación de reportes necesarios para planeación y atención de necesidades de otras áreas. Así como el hecho de que los alumnos cambien actualmente de Genero y de unidad académica, y que el numero de boleta se use como identificador, dificulta el cambio de boleta o la consistencia de la información. 
	\item[Propuesta:] A la par del desarrollo del Calmécac se propone revisar los datos cambiantes y potencialmente cambiantes del alumno y despojar al Numero de boleta de dicha información, así como enriquecer el Calmécac con los reportes que las Unidades académicas requieren para su operación. 
	\item[Impacto:] Alto.
	\item[Alcance:] Dentro del ámbito del proyecto Calmécac la definición y generación de reportes.
\end{AreaOportunidad}



%------------------------------------------------
\section{Proceso de Reinscripciones}
%- - - - - - - - - - - - - - - - - - - - - - - - 
\begin{AreaOportunidad}{PR-AO1}
	\item[Área:] \refElem{DAE}.
	\item[Procesos:] Reinscripción.
	\item[Problema:] El SAES actualmente no distingue entre modalidad mixta y presencial.
	\item[Impacto:] Bajo.
	\item[Alcance:] Se propone que el Calmécac distinga entre las dos modalidades.
\end{AreaOportunidad}

%- - - - - - - - - - - - - - - - - - - - - - - -

\begin{AreaOportunidad}{PR-AO2}
	\item[Área:] \refElem{DAE}.
	\item[Procesos:] Reinscripción.
	\item[Problema:] El SAES fue diseñado sin considerar que un programa académico puede tener varias especialidades.
	\item[Impacto:] Medio.
	\item[Alcance:]Se propone que el Calmécac considere que un programa académico puede tener varias especialidades y que valide que un alumno únicamente pueda inscribir unidades de aprendizaje de la especialidad registrada.
\end{AreaOportunidad}

%- - - - - - - - - - - - - - - - - - - - - - - - - 


\begin{AreaOportunidad}{PR-AO3}
	\item[Área:] \refElem{DAE}.
	\item[Procesos:] Reinscripción.
	\item[Problema:] El módulo de dictámenes que considera el SAES no satisface las necesidades de las reinscripciones.
	\item[Impacto:] Alto.
	\item[Alcance:]Construir un módulo para la gestión de dictámenes que ayude a conocer la situación escolar del alumno, a fin de determinar las restricciones a aplicar para su reinscripción con base en el dictamen.	
\end{AreaOportunidad}

%- - - - - - - - - - - - - - - - - - - - - - - - - -

\begin{AreaOportunidad}{PR-AO4}
	\item[Área:] \refElem{DAE}.
	\item[Procesos:] Reinscripción.
	\item[Problema:] El SAES no permite identificar cuando un alumno no tiene una materia desfasada en su programa académico.
	\item[Impacto:] Alto.
	\item[Alcance:]Se propone que el Calmécac Identifique las materias desfasadas y limite la reinscripción de acuerdo al reglamento general de estudios.	
\end{AreaOportunidad}

%- - - - - - - - - - - - - - - - - - - - - - - - - - -

\begin{AreaOportunidad}{PR-AO5}
	\item[Área:] \refElem{DAE}.
	\item[Procesos:] Reinscripción.
	\item[Problema:] Es complicado acreditar en el SAES una materia por saberes previamente adquiridos, debido a que cuando un alumno acredita una unidad de aprendizaje por este medio el SAES, se tiene que realizar la baja de forma manual.
	\item[Impacto:] Bajo.
	\item[Alcance:]Se propone que el Calmécac permita la gestión de registro de calificaciones de unidades de aprendizaje previamente adquiridos.
\end{AreaOportunidad}

%- - - - - - - - - - - - - - - - - - - - - - - - - - -

\begin{AreaOportunidad}{PR-AO6}
	\item[Área de Oportunidad:] PR-AO6
	\item[Área:] \refElem{DAE}.
	\item[Procesos:] Reinscripción.
	\item[Problema:] Plataforma en Modalidad Virtual no tiene vinculación con SAES.
	\item[Impacto:] Alto.
	\item[Alcance:] Buscar un mecanismo de comunicación entre la modalidad virtual y la presencial.
\end{AreaOportunidad}

%- - - - - - - - - - - - - - - - - - - - - - - - - - - -

\begin{AreaOportunidad}{PR-AO7}
	\item[Área:] \refElem{DAE}.
	\item[Procesos:] Reinscripción.
	\item[Problema:] El SAES está desarrollado para modalidad escolarizada, en cuestión de los horarios, no muestra horarios para los programas académicos en modalidad mixta.
	\item[Impacto:] Alto.
	\item[Alcance:] Distinguir entre la modalidad mixta y presencial, así como en que modalidad el estudiante aprueba sus materias.
\end{AreaOportunidad}

%- - - - - - - - - - - - - - - - - - - - - - - - - - - - 

%------------------------------------------------
\section{Proceso de Registro de Evaluaciones}
%- - - - - - - - - - - - - - - - - - - - - - - -

\begin{AreaOportunidad}{PRE-AO1}
	\item[Área:] \refElem{UnidadAcademica} y \refElem{DES}.
	\item[Procesos:] Registro de Evaluaciones.
	\item[Problema:] Actualmente los procesos que se tienen para el registro de calificaciones presentan vulnerabilidades. Existen huecos de seguridad a lo largo de todo el proceso.
	\item[Área de Oportunidad:] Fortalecer dichos procesos implementando la firma digital como mecanismo de seguridad para verificar la autenticidad y garantizar el no repudio de cualquier calificación, así como un folio digital para el subproceso de modificación de calificaciones como un identificador único en el Instituto Politécnico Nacional para el acta que se generará.
	\item[Impacto:] Alto.
	\item[Alcance:] Dentro del CALMÉCAC se integrará un mecanismo de seguridad (firma electrónica) que aumente los servicios de integridad y permita no repudio.
\end{AreaOportunidad}

%- - - - - - - - - - - - - - - - - - - - - - - -

\begin{AreaOportunidad}{PRE-AO2}
	\item[Área:] \refElem{UnidadAcademica}.
	\item[Procesos:] Registro de Evaluaciones.
	\item[Problema:] El mantener un Kárdex físico presenta un problema cuando la población estudiantil es demasiado grande, así como de salud después de varios años.
	\item[Área de Oportunidad:]El Kárdex se puede implementará de manera digital la actualización podría ser automática en cuanto el profesor subiera la calificación y la cantidad de alumnos no presentaría ningún problema.
	\item[Impacto:] Medio.
	\item[Alcance:] Con la implementación de diversos mecanismos de seguridad que tendrá el CALMÉCAC se puede almacenar de manera digital  el kárdex para cada unidad académica.
\end{AreaOportunidad}

%- - - - - - - - - - - - - - - - - - - - - - - -

\begin{AreaOportunidad}{PRE-AO3}
	\item[Área:] \refElem{UnidadAcademica}.
	\item[Procesos:] Registro de Evaluaciones.
	\item[Problema:] El sistema no diferencia entre las diversas modalidades de estudios.
	\item[Área de Oportunidad:]Diferenciar las diversas modalidades escolares cuando se realicen las evaluaciones ordinarias.
	\item[Impacto:] Alto.
	\item[Alcance:] El CALMÉCAC podrá distinguir entre las diferentes modalidades de estudios.
\end{AreaOportunidad}

%- - - - - - - - - - - - - - - - - - - - - - -

\begin{AreaOportunidad}{PRE-AO4}
	\item[Área:] \refElem{UnidadAcademica} y \refElem{DES}.
	\item[Procesos:] Registro de Evaluaciones.
	\item[Problema:] El proceso de inscripción a los Exámenes a Título de Suficiencia se hace de manera presencial por lo que requiere de bastante tiempo y personal para poder cubrir la demanda de estos.
	\item[Área de Oportunidad:] El registro de los Exámenes a Título de Suficiencia puede optimizarse. Si se hiciera de manera digital por medio de un sistema que indicará qué alumnos son candidatos a realizar dicho examen.
	\item[Impacto:] Medio.
	\item[Alcance:] El CALMÉCAC facilitará el registro de los alumnos a los Exámenes a Título de Suficiencia gestionando qué alumnos son candidatos a realizar dicho examen.
\end{AreaOportunidad}

%- - - - - - - - - - - - - - - - - - - - - - - -

\begin{AreaOportunidad}{PRE-AO5}
	\item[Área:] \refElem{UnidadAcademica} .
	\item[Procesos:] Registro de Evaluaciones.
	\item[Problema:] Falta validación de inicio de evaluaciones y opción a corregir las fechas de inicio y fin de éstas.
	\item[Área de Oportunidad:] El inicio del proceso de registrar las evaluaciones podría ir ligado al calendario escolar de la modalidad escolar a la que pertenece.  
	\item[Impacto:] Medio.
	\item[Alcance:] El CALMÉCAC facilitara el manejo de periodos de evaluación.
\end{AreaOportunidad}



%================================================
\section{Procesos fortalecidos}

	En los siguientes capítulos se muestran los proceso fortalecidos. En ellos se implementan las Áreas de oportunidad de esta sección detallando los cambios de cada proceso.

%%------------------------------------------------
%\subsection{}
%
%%------------------------------------------------
%\subsection{}

