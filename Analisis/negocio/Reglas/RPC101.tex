\begin{BusinessRule}{RPC101}{CONDICIONES DE UN REFUGIO.}{
		Regla de inferencia de un hecho.
		% Otras opciones para tipo: 
		% - Regla de integridad referencial o estructural. 
		% - Regla de operación, (calcular o determinar un valor.).
		% - Regla de inferencia de un hecho.
	}{
		Habilitadora. 
		% Otras opciones para clase: Habilitadora, Cronometrada, Ejecutive.
	}{
		Controla la operación. % Otras opciones para nivel: Controla la operación, Influencia (dirige) la operación.
	}
	\BRItem[Descripción:] Indentificar una instalacion como albergue temporal depende de algunos factores elementales, lo cual le permitirá garantizar una oportuna y eficaz protección a la ciudadania ante la inminecia de un peligro .Como las Siguientes\\1.-LAS CONDICIONES DE LA INSTALACIÓN: \\1.1.-influye en el tipo de construcción y estado de mantenimiento constructivo de la instalación.\\1.2.-Disponibilidad y condiciones de los servicios sanitarios y para lavado de ropa elemental necesidad, fundamentalmente de niños.\\1.3.-Numero de personas que puede refugiar partiendo de las normas que se establecen en estos lineamientos. \\2.-DISPONIBILIDAD DE SERVICIOS:\\2.1.-En la disponibilidad de lso servicios juega un ppel muy importante el volumen de agua para el consumo humano, laproporción adecuada de las instalaciones sanitarias y su estado, así como la distribución final de residuales líquidos y el tratamiento a los desechos sólidos. \\3.-CAPACIDAD DE REFUGIO.\\3.1.-La Organización Mundial de la Salud (OMS) recomienda que para alojamiento de emergencia se de debe garantizar como norma 3.5 metros cuadrados por persona , no incluyendo en ello areas recreatvas , cocinas , baños , comedor y alamacenes.
	%\BRItem[Ejemplo positivo:] 
	
	%\BRItem[Ejemplo negativo:] 
	
	\BRItem[Referenciado por:] LA GACETA OFICIAL DE LA CIUDAD DE MÉXICO , SECRETARIA DE PROTECCION CIVIL"NORMA TÉCNICA COMPLEMENTARIA NTCPC-007-ALERTAMIENTO SÍSMICO-2017"
\end{BusinessRule}