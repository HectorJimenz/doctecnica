\begin{BusinessRule}{RPC109}{GESTION INTEGRAL DE RIESGOS}{
		Regla de inferencia de un hecho..
		% Otras opciones para tipo: 
		% - Regla de integridad referencial o estructural. 
		% - Regla de operación, (calcular o determinar un valor.).
		% - Regla de inferencia de un hecho.
	}{
		Habilitadora. 
		% Otras opciones para clase: Habilitadora, Cronometrada, Ejecutive.
	}{
		Controla la operación. % Otras opciones para nivel: Controla la operación, Influencia (dirige) la operación.
	}
	\BRItem[Descripción:] La Gestión Integral de Riesgos deberá contribuir al conocimiento integral del Riesgo para el desarrollo de las ideas y principios que perfilarán la toma de decisiones y, en general, las políticas públicas, estrategias y procedimientos encaminados a la reducción del mismo.\\1.- La planeación que defina la visión, objetivos y condiciones necesarias para construir un esquema de Gestión Integral de Riesgos.\\ \\2.- El mejoramiento del nivel y calidad de vida de la población urbana y rural, a través de los programas y estrategias dirigidas al fortalecimiento de los instrumentos de organización y funcionamiento de las instituciones de Protección Civil, así como los planes de desarrollo, teniendo como base un enfoque estratégico y proactivo y las acciones para prevenir y mitigar los Riesgos, apoyadas en el Atlas Nacional de Riesgo, y en los Atlas Estatales y Municipales de Riesgos y, en su caso, en aquellas actividades tendientes a la atención de Emergencias y la Reconstrucción.
	%\BRItem[Ejemplo positivo:] 
	
	%\BRItem[Ejemplo negativo:] 
	
	\BRItem[Referenciado por:] Reglamento de la Ley General de Protección Civil
\end{BusinessRule}