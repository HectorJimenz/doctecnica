\begin{BusinessRule}{RPC106}{AUTOCUIDADO}{
		Regla de inferencia de un hecho.
		% Otras opciones para tipo: 
		% - Regla de integridad referencial o estructural. 
		% - Regla de operación, (calcular o determinar un valor.).
		% - Regla de inferencia de un hecho.
	}{
		Habilitadora. 
		% Otras opciones para clase: Habilitadora, Cronometrada, Ejecutive.
	}{
		Controla la operación. % Otras opciones para nivel: Controla la operación, Influencia (dirige) la operación.
	}
	\BRItem[Descripción:] Las acciones destinadas a la Reducción de Riesgos en sus aspectos preventivos, a favor de sí mismo, de la familia y de la comunidad a la que se pertenece.
	\\Saber cómo actuar en un sismo nos ayuda a salvar la vida y la de los demás. Por eso, es imprescindible tomar los recaudos necesarios y saber qué debemos hacer en caso de que suceda, ya sea que estemos en nuestra vivienda, trabajo o lugar de esparcimiento.\\En este sentido, es importante tener presentes las recomendaciones para actuar ante cualquier sismo o situación de riesgo.\\Es primordial que las familias armen un plan, se organicen y sepan cómo van a actuar y qué rol va a tener cada uno ante la emergencia. Luego, hay que tener preparada la “mochila” o “kit” de emergencia, con lo mínimo e indispensable para estos casos. Estas recomendaciones pertenecen al Plan de Acción Familiar que se trabaja desde el Gobierno.
	%\BRItem[Ejemplo positivo:] 
	
	%\BRItem[Ejemplo negativo:] 
	
	\BRItem[Referenciado por:] Reglamento de la Ley General de Protección Civil
\end{BusinessRule}