\begin{BusinessRule}{RPC110}{GRUPOS VOLUNTARIOS}{
		Regla de operación, (calcular o determinar un valor.).
		% Otras opciones para tipo: 
		% - Regla de integridad referencial o estructural. 
		% - Regla de operación, (calcular o determinar un valor.).
		% - Regla de inferencia de un hecho.
	}{
		Habilitadora. 
		% Otras opciones para clase: Habilitadora, Cronometrada, Ejecutive.
	}{
		Controla la operación. % Otras opciones para nivel: Controla la operación, Influencia (dirige) la operación.
	}
	\BRItem[Descripción:]  El registro de Grupos Voluntarios a que se refiere el artículo 51 de la Ley, constituye uno de los elementos para lograr la coordinación entre el gobierno y la sociedad, que permita fomentar la participación social referida en la Ley.\\Un Grupo Voluntario tendrá el carácter de nacional cuando pueda atender Emergencias en cualquier parte del país y el carácter de regional cuando atienda a dos o más entidades federativas colindantes.\\Para inscribirse como voluntario para la remoción de escombros, es necesario inscribirse en el Escuadrón de Rescate y Urgencias Médicas (ERUM) de la Secretaría de Seguridad Pública, en Chimalpopoca, colonia Centro.\\Para ser voluntario es necesario ser mayor de edad, no tener enfermedades infectocontagiosas o respiratorias, ni enfermedades degenerativas crónicas, según le dijo a CNN en Español Alejandro Villegas, subdirector de Capacitación y Vinculación del ERUM.\\

Una vez son seleccionados, a los voluntarios se les equipa con herramientas de seguridad como cascos, guantes, tapabocas, entre otros, y forman grupos para ser desplazados hacia los lugares más afectados
	%\BRItem[Ejemplo positivo:] 
	
	%\BRItem[Ejemplo negativo:] 
	
	\BRItem[Referenciado por:] Reglamento de la Ley General de Protección Civil
\end{BusinessRule}