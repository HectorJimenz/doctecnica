\begin{BusinessRule}{RPC102}{CONDICIONES DE UN ALBERGUE.}{
		Regla de inferencia de un hecho.
		% Otras opciones para tipo: 
		% - Regla de integridad referencial o estructural. 
		% - Regla de operación, (calcular o determinar un valor.).
		% - Regla de inferencia de un hecho.
	}{
		Habilitadora. 
		% Otras opciones para clase: Habilitadora, Cronometrada, Ejecutive.
	}{
		Controla la operación. % Otras opciones para nivel: Controla la operación, Influencia (dirige) la operación.
	}
	\BRItem[Descripción:] Para la evalucón de los albergues existe un grupo llamado "COMISIÓN DE ALBERGUE" nombrado por el director del organismo a la cual pertenece, son quienes revisan periodicamente y durante la fase informativa las condicones generales y recursos.\\De acuerdo a la COMISIÓN DE ALBERGES.\\1.-Debe contar con alojamiento y protección.\\2.-Alimentación.\\3.-Vesturio.\\4.-Recreación y esparcimiento.\\5.- Atención medica.\\6.- Seguridad.\\7.-Higiene y Saneamiento.\\8.-La Organización Mundial de la Salud (OMS) recomienda que para alojamiento de emergencia se de debe garantizar como norma 3.5 metros cuadrados por persona , no incluyendo en ello areas recreatvas , cocinas , baños , comedor y alamacenes.\\EXISTEN TRES TIPOS DE ALBERGUE:\\1.- Albergue familiar.\\2.-Albergue comunitario en instalación cerrada.\\3.-Albergue tipo cabaña. 
	%\BRItem[Ejemplo positivo:] 
	
	%\BRItem[Ejemplo negativo:] 
	
	\BRItem[Referenciado por:] LA GACETA OFICIAL DE LA CIUDAD DE MÉXICO , SECRETARIA DE PROTECCION CIVIL"NORMA TÉCNICA COMPLEMENTARIA NTCPC-007-ALERTAMIENTO SÍSMICO-2017"
\end{BusinessRule}