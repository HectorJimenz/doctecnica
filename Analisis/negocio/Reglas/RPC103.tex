\begin{BusinessRule}{RPC103}{CONDICIONES DE UN CENTRO DE ACOPIO.}{
		Regla de inferencia de un hecho.
		% Otras opciones para tipo: 
		% - Regla de integridad referencial o estructural. 
		% - Regla de operación, (calcular o determinar un valor.).
		% - Regla de inferencia de un hecho.
	}{
		Habilitadora. 
		% Otras opciones para clase: Habilitadora, Cronometrada, Ejecutive.
	}{
		Controla la operación. % Otras opciones para nivel: Controla la operación, Influencia (dirige) la operación.
	}
	\BRItem[Descripción:] Es un espacio físico, instalados por las organizaciones e instancias públicas, privadas y sociales para recibir la ayuda humanitaria, cuando ésta es requerida oficialmente.\\La apertura de un centro de acopio se efectúa cuando la autoridad administradora de una emergencia lo determina, previa evaluación de los daños y recursos existentes para la atención de la población afectada o damnificada por una emergencia. \\Los centros de acopio deberán recibir insumos de acuerdo con lo señalado en las sugerencias para los donantes. Estableciendo parámetros como, tipo de producto, peso, volumen y dimensionar las necesidades y tipo de transporte requerido para su traslado.\\A continuación se dan algunas consideraciones que deberán reunir los artículos:\\1.-Ropa , zapatos y vestimenta.\\2.-Medicamentos.\\3.-Material de curación.\\4.-Alimentos.\\5.-Agua.\\6.-Sangre.\\7.-Blancos.
	%\BRItem[Ejemplo positivo:] 
	
	%\BRItem[Ejemplo negativo:] 
	
	\BRItem[Referenciado por:] LA GACETA OFICIAL DE LA CIUDAD DE MÉXICO , SECRETARIA DE PROTECCION CIVIL"NORMA TÉCNICA COMPLEMENTARIA NTCPC-007-ALERTAMIENTO SÍSMICO-2017"
\end{BusinessRule}