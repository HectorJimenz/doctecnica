\begin{BusinessRule}{RPC105}{ALERTA SISMICA}{
		Regla de inferencia de un hecho.
		% Otras opciones para tipo: 
		% - Regla de integridad referencial o estructural. 
		% - Regla de operación, (calcular o determinar un valor.).
		% - Regla de inferencia de un hecho.
	}{
		Cronometrada. 
		% Otras opciones para clase: Habilitadora, Cronometrada, Ejecutive.
	}{
		Controla la operación. % Otras opciones para nivel: Controla la operación, Influencia (dirige) la operación.
	}
	\BRItem[Descripción:] El aviso de la proximidad de un Fenómeno Antropogénico o Natural Perturbador o el incremento del Riesgo Asociado al mismo.\\El sistema se basa en el principio de las ondas sísmicas superficiales, las cuales son consideradas como potencialmente dañinas; éstas viajan de entre 3.5 y 4.0 kilómetros por segundo, lo que significa que tardan entre 75 y 85 segundos en viajar de Guerrero a la Ciudad de México, sin embargo el tiempo en que llega la señal puede variar dependiendo de la distancia del epicentro, la produndidad y la intensidad del sismo.\\Oficialmente, la alarma se activa con sismos de magnitudes cercanas a los 6 grados y se transmite en los 8 mil 200 altavoces distribuidos en las 16 delegaciones de la Ciudad de México.
A partir de entonces el Centro de Atención a Emergencias y Protección Ciudadana de la Ciudad de México (CAEPCCM), se integró a la difusión de avisos de Alerta Sísmica.
	%\BRItem[Ejemplo positivo:] 
	
	%\BRItem[Ejemplo negativo:] 
	
	\BRItem[Referenciado por:] Reglamento de la Ley General de Protección Civil
\end{BusinessRule}