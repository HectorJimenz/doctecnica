\begin{BusinessRule}{RPC118}{CULTURA DE PROTECCIÓN CIVIL}{
		Regla de operación, (calcular o determinar un valor.).
		% Otras opciones para tipo: 
		% - Regla de integridad referencial o estructural. 
		% - Regla de operación, (calcular o determinar un valor.).
		% - Regla de inferencia de un hecho.
	}{
		Habilitadora. 
		% Otras opciones para clase: Habilitadora, Cronometrada, Ejecutive.
	}{
		Controla la operación. % Otras opciones para nivel: Controla la operación, Influencia (dirige) la operación.
	}
	\BRItem[Descripción:] Para que la sociedad participe en la planeación y supervisión de la Protección Civil, el mecanismo idóneo en lo que corresponde al Gobierno Federal, es el Consejo Consultivo del Consejo Nacional.\\Las dependencias y entidades de la Administración Pública Federal promoverán el acceso a la información actualizada sobre los Peligros, Vulnerabilidades y Riesgos de origen natural y antropogénicos, a través de los medios de difusión que estén a su alcance.\\Los programas y las campañas de difusión de la Cultura de Protección Civil, dirigidos a la población, deberán dar a conocer de forma clara, los mecanismos de Prevención y Autoprotección.\\Con el fin de fomentar la Cultura de Protección Civil, la Coordinación Nacional promoverá ante las autoridades educativas competentes que los planes y programas de estudio oficiales aplicables y obligatorios en la República Mexicana, en todos los niveles educativos, incluyan contenidos temáticos de Protección Civil y de la Gestión Integral de Riesgos.
	%\BRItem[Ejemplo positivo:] 
	
	%\BRItem[Ejemplo negativo:] 
	
	\BRItem[Referenciado por:] Reglamento de la Ley General de Protección Civil
\end{BusinessRule}