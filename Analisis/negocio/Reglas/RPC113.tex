\begin{BusinessRule}{RPC113}{SISTEMA NACIONAL EN LOS SISTEMAS DE ALARMAS TEMPRANAS}{
		Regla de inferencia de un hecho..
		% Otras opciones para tipo: 
		% - Regla de integridad referencial o estructural. 
		% - Regla de operación, (calcular o determinar un valor.).
		% - Regla de inferencia de un hecho.
	}{
		Habilitadora. 
		% Otras opciones para clase: Habilitadora, Cronometrada, Ejecutive.
	}{
		Controla la operación. % Otras opciones para nivel: Controla la operación, Influencia (dirige) la operación.
	}
	\BRItem[Descripción:] A la Coordinación Nacional, en su carácter de responsable de la coordinación ejecutiva del Sistema Nacional, le compete promover y coordinar entre los integrantes del Sistema Nacional, la implementación de los Sistemas de Monitoreo y Sistemas de Alertas Tempranas, así como incorporar a dichos sistemas los esfuerzos de otras redes de monitoreo públicas de las entidades federativas o del sector privado.\\La Coordinación Nacional fomentará y, en su caso, establecerá mecanismos de colaboración con los integrantes del Sistema Nacional que lleven a cabo el monitoreo de fenómenos naturales, con el objeto de intercambiar información relacionada con los Sistemas de Alerta Temprana.\\Las dependencias y entidades de la Administración Pública Federal, que realicen el monitoreo de los fenómenos naturales para operar Sistemas de Alerta Temprana, deberán prever en sus presupuestos los recursos necesarios para garantizar el óptimo funcionamiento de dichos Sistemas, así como la sostenibilidad de los mismos.
	%\BRItem[Ejemplo positivo:] 
	
	%\BRItem[Ejemplo negativo:] 
	
	\BRItem[Referenciado por:] Reglamento de la Ley General de Protección Civil
\end{BusinessRule}