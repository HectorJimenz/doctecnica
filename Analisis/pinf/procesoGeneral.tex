
% Descripción general del proceso
%-----------------------------------------------
\begin{procesoGeneral}{PA-GINF}{Proceso General de Gestión de Infraestructura} {
		
		%-------------------------------------------
		%Resumen
		
		El proceso de Gestión de Infraestructura se encarga de gestionar los espacios en los cuales se impartirán las clases de las distintas unidades de aprendizaje\footnote{ver \refElem{UnidadDeAprendizaje}} de los programas académicos\footnote{ver \refElem{ProgramaAcademico}}. Esta información conformará un catálogo general que estará a disposición de todo el instituto.
					
		%-------------------------------------------
		%Diagrama del proceso
		\noindent La figura \cdtRefImg{pGeneral:PG-GINF}{Diagrama  General del Proceso de Gestion de Infraestructura} muestra las actividades que se realizan para llevar a cabo el proceso descrito anteriormente.
		
		\Pfig[0.8]{pinf/imagenes/macroInfraestructura.png}{pGeneral:PG-GINF}{Diagrama General del Proceso de Gestion de Infraestructura}
	}{PG-GINF}

\end{procesoGeneral}

%========================================================
%Descripción de tareas
%-----------------------------------------------
\begin{PDescripcion}

%Actor: SAEV2.0
\Ppaso \textbf{Registro de Horarios}

\begin{enumerate}
	
	\Ppaso[\PSubProceso] \cdtLabelTask{PP-IR3.2:SAEV2.0}{ \textbf{Gestión de Edificios.}} Este proceso permitirá a cada \refElem{UnidadAcademica} gestionar los edificios que forman parte de sí misma.

	%Subproceso 2		
	\Ppaso[\PSubProceso] \cdtLabelTask{PP-IR3.1:SAEV2.0}{ \textbf{Gestión de Unidades Externas.}} Este proceso permitirá a cada \refElem{UnidadAcademica} gestionar los edificios de las entidades que no forman parte de la institución, pero en las cuales se tiene permitido impartir clases para algunas unidades de aprendizaje que asi lo requieren, como el caso de hospitales.
		
	\Ppaso[\PSubProceso] \cdtLabelTask{PP-IR3.3:SAEV2.0}{ \textbf{Gestión de espacios.}} Este proceso permitirá a las unidades académicas\footnote{ver \refElem{UnidadAcademica}} gestionar cada uno de los espacios de cada edificio en los cuales se impartirán clases.
	
	
\end{enumerate}

%\Ppaso[\PSubProceso]\cdtLabelTask{PP-IR3.2:SAEV2.0} \textbf{Gestión de Edificios.} Este proceso permite a registrar las necesidades de edificios de las unidades académicas.

%	
%	%Actor: SAEV2.0
%	\Ppaso \textbf{SAEV2.0}
%	
%	\begin{enumerate}
%		%Subproceso 1
%		\Ppaso[\PSubProceso] \cdtLabelTask{PP-IR3.2:SAEV2.0}{ \textbf{Gestión de Edificios.} Este proceso permitirá a cada unidad académica gestionar los edificios que son forman parte de sí misma.
%		
%		%Subproceso 2
%		%\Ppaso[\PSubProceso] \cdtLabelTask{PP-IR3.1:SAEV2.0}{ \textbf{Gestión de Unidades Externas.} Este proceso permitirá a cada unidad académica gestionar los edificios de las entidades que no forman parte de la institución, pero que en las cuales se tiene permitido impartir clases, como el caso de hospitales.
%	
%		%\Ppaso[\PSubProceso] \cdtLabelTask{PP-IR3.3:SAEV2.0}{ \textbf{Gestión de espacios.} Este proceso permitirá a las unidades académicas gestionar cada uno de los espacios de cada edificio en los cuales se impartirán clases.
%			
%	\end{enumerate}
\end{PDescripcion}