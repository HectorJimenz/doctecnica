\begin{Proceso}{PG.X}{Nombre del proceso general}
	{
		Anote la descripción del proceso general, resumiendo sus objetivos, las áreas/personas involucradas, y los procesos contenidos.
	}
	{PG.X}
	\UCccsection{Administración del proceso}
	\UCccitem{Versión}{1}
	\UCccitem{Autor}{Nombre completo del responsable del proceso}
	\UCccitem{Evaluador}{Nombre completo del evaluador}
	\UCccitem{Prioridad}{Alta/Media/Baja}
	\UCccitem{Estatus}{Terminado/Corrección/Aprobado}
	\UCccitem{Complejidad}{Alta/Media/Baja}
	\UCccitem{Volatilidad}{Alta/Media/Baja}
	\UCccitem{Madurez}{Alta/Media/Baja}
	\UCccsection{Control de cambios}
	\UCccitem{Versión 0}{
			\begin{UClist}
				\RCitem{ PC1}{Corregir la ortografía}{\DONE}
				%\TODO es para solicitar un cambio \TOCHK Es para informar que se atendió el TODO(ya hizo las correcciones),\DONE Es para indicar que el e valuador reviso los cambios.
			\end{UClist}
	}
	\UCitem{Participantes}
	{
		 Liste los participantes en el proceso, ya sean: áreas, organos colegiados o individuos, utilice el cmnado \refActor{refActor}.
	}
	\UCitem{Objetivo}
	{
		Escriba un resumen a manera de objetivo (Que-Caracterisitica-para que) que englobe las responsabilidades relacionadas con el proceso y los problemas que resuelve.
	}
	\UCitem{Soporte}
	{
		El presente macro proceso está definido a fin de dar soporte a las responsabilidades: liste las responsabilidades asociadas al proceso.
	}

	\UCitem{Interrelación con otros procesos}
	{
		Liste los procesos con que se enlaza la operación del proceso actual.
	}
	\UCitem{Entradas}{
		Liste los datos, formatos o insumos que se requieren como entradas a lo largo de este procesos.
	}
	\UCitem{Salidas}{
		Liste los datos, formatos o insumos que se requieren como salidas o productos a lo largo de este procesos.
	}
	\UCitem{Precondiciones}{
		Liste las actividades, productos o condiciones que deben ocurrir o cumplirse antes de iniciar el proceso.
	}
	\UCitem{Postcondiciones}{
		Liste los productos o condiciones que se logran o cambian al terminar el proceso de forma correcta.
	}
	\UCitem{Frecuencia}{
		Periódico, programado, por evento.
		% Periódico: Cada cierto tiempo: diario, semanal, anual, etc.
		% Programado: Alguien en algún momento establece la fecha.
		% Eventual: Cada que ocurre un evento que no se puede prever ni programar.
	}
	\UCitem{Tipo}{
		Mejora Continua / Operación / Soporte
	}
	% Operacion: Proceso asociado a las actividades propias de la operación del RENIECYT.
	% Mejora continua: Porcesos asociados a actividades de mejora continua del proceso actual.
	% Soporte: Procesos asociados a actividades indirectas necesarias para operar el RENIECYT.
	\UCitem{Macroproceso}{
		Referencia al macroprocesos en donde se encuentra el proceso.
	}
	\UCitem{Problemas que resuelve}{
		Liste los problemas que este nuevo proceso pretende resolver o atacar. 
	}
	
\end{Proceso}

%Los procesos definidos para este porceso general se muestran en la figura\ref{fig:procesoGral1}

%\begin{figure}[htbp!]
	%\begin{center}
		%\includegraphics[width=.8\textwidth]{images/procesoGral1}
		%\caption{Descripción del proceso.}
		%\label{fig:macroproceso1}
	%\end{center}
%\end{figure}


%Descripción de procesos
\begin{PDescripcion}
	\Ppaso \textbf{Nombre del Proceso:} Descripción del proceso...
\end{PDescripcion}


%Factores criticos
\begin{FCDescripcion}
	\FCpaso Listar los factores críticos en este proceso
\end{FCDescripcion}
