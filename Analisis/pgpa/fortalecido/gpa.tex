\begin{Proceso}{PE-GPA}{Gestión de P A} %Antes PG4.7
	{
		resumen
	}
	{PA1}
	\UCccsection{Administración del proceso}
	\UCccitem{Versión}{1}
	\UCccitem{Autor}{José Rogelio Hernández López}
	\UCccitem{Evaluador}{Nombre completo del evaluador}
	\UCccitem{Prioridad}{Media}
	\UCccitem{Estatus}{Terminado}
	\UCccitem{Complejidad}{Baja}
	\UCccitem{Volatilidad}{Media}
	\UCccitem{Madurez}{Media}
	\UCccsection{Control de cambios}
	\UCccitem{Versión 0}{
			\begin{UClist}
				\RCitem{ PC1}{Corregir la ortografía}{\DONE}
				%\TODO es para solicitar un cambio ppp\TOCHK Es para informar que se atendió el TODO(ya hizo las correcciones),\DONE Es para indicar que el evaluador reviso los cambios.
			\end{UClist}
	}

	\UCitem{Objetivo}
	{
		\begin{UClist}
			\UCli Establecer los periodos y mecanismos necesarios para capacitar a los \refActor{Dictaminadores}.
		\end{UClist}
	}

	\UCitem{Proveedores}
	{
		\begin{UClist}
		    \UCli \refElem{DES}.
        \end{UClist}
	}

	\UCitem{Consumidores}
	{
		\begin{UClist}
		\UCli \refElem{DES}.
		\end{UClist}
	}


	\UCitem{Entradas}
    {
		\begin{UClist}
            \UCli Las propuestas y mecanismos para la capacitación de los \refActor{Dictaminadores}.
        \end{UClist}
		%Liste los datos, formatos o insumos que se requieren como entradas a lo largo de este procesos.
	}
	\UCitem{Salidas}
    {
		\begin{UClist}
            \UCli \refActor{Dictaminadores} capacitados, de acuerdo al proceso del RENIECYT.
        \end{UClist}
		%Liste los datos, formatos o insumos que se requieren como salidas o productos a lo largo de este procesos.
	}
	\UCitem{Precondiciones}
    {
		\begin{UClist}
            \UCli Existir necesidades de cambio de capacitación a los \refActor{Dictaminadores}.
            \UCli Existir fechas para llevar a cabo la capacitación.
        \end{UClist}
%		Liste las actividades, productos o condiciones que deben ocurrir o cumplirse antes de iniciar el proceso.
	}
	\UCitem{Postcondiciones}
    {
		\begin{UClist}
            \UCli Mejorar las aptitudes y capacidades de los \refActor{Dictaminadores}.
            \UCli Mantener actualizado al personal encargado de evaluar las Solicitudes de Inscripción.
        \end{UClist}
%		Liste los productos o condiciones que se logran o cambian al terminar el proceso de forma correcta.
	}

	\UCitem{Reglas de negocio}
	{
	\begin{UClist}
		\UCli Mejorar las aptitudes y capacidades de los \refActor{Dictaminadores}.
		\UCli Mantener actualizado al personal encargado de evaluar las Solicitudes de Inscripción.
	\end{UClist}
	%		Liste los productos o condiciones que se logran o cambian al terminar el proceso de forma correcta.
	}

	\UCitem{Problemas que resuelve}
    {
		\begin{UClist}
            \UCli \hyperlink{P7}{Ambigüedad en los procesos}
                   \end{UClist}
%		Liste los problemas que este nuevo proceso pretende resolver o atacar.
	}
\end{Proceso}

	%Los procesos definidos para este porceso general se muestran en la figura\ref{fig:procesoGral1}

%\begin{figure}[htbp!]
	%\begin{center}
		%\includegraphics[width=.8\textwidth]{images/procesoGral1}
		%\caption{Descripción del proceso.}
		%\label{fig:macroproceso1}
	%\end{center}
%\end{figure}
