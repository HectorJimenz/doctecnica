\chapter{Reportes}
\label{appendix:Reportes}

El siguiente capítulo contiene los reportes generados a lo largo del proceso RENIECYT.\\

% \newpage
% %---------------------------------------------------------
% \section{Listado de Reportes}
% \label{appendix:Reportes:PeopleSoft:ListadoReportes}

\noindent El sistema de \refActor{PeopleSoft} genera los reportes listados a continuación. Los reportes que se encuentran sombreados, son los reportes que actualmente utiliza el módulo de administración del RENIECYT.\\

\begin{tabular}{ m{.46\textwidth} m{.49\textwidth}  }%
	\rowcolor{gray1} {\bf Campo} &  {\bf Descripción} \\ \hline \hline

	\rowcolor{gray1}CYT\_APOYO\_A\_RENIECYT & Apoyos que ha recibido una emp. \\
	CYT\_REG\_NO\_CONCLUIDOS & Reporte\_Reg\_No\_Concluidos.\\
	CYT\_REPORTE\_ESTADOS &  Reporte Agrupado por Estados.\\
	CYT\_REPORTE\_GENERAL\_FERIA &  Reporte General Feria Posgrado. \\
	\rowcolor{gray1}CYT\_RNCYT2\_CAMBIO\_RLEGAL &  Reporte Cambios de R Legal. \\
	CYT\_RNCYT2\_EMPRESAS\_VIGENTES2 &  empresas vigentes actual. \\
	\rowcolor{gray1}CYT\_RNCYT2\_HISTORIAL\_X\_SOLICIT &  Rncyt - Historial por Solicit.\\
	CYT\_RNCYT2\_HISTORIAL\_X\_STATUS &  Rncyt - Historial por Status. \\
	CYT\_RNCYT2\_HIST\_FECHA\_COORD &  Rncyt - Hist\_X\_Fecha\_y\_Coord. \\
	CYT\_RNCYT2\_HIST\_FECHA\_EVALUDOR &  Rncyt - Hist\_X\_Fecha\_y\_Eval. \\
	CYT\_RNCYT2\_HIST\_X\_STATUS\_FECHA &  Rncyt - Hist. X Status y Fecha. \\
	\rowcolor{gray1}CYT\_RNCYT2\_HIST\_X\_ST\_RG\_FECHA &  Rncyt-Hist. X Status\_Rango\_Fec. \\
	\rowcolor{gray1}CYT\_RNCYT2\_HST\_X\_RG\_FCH\_COOR &  Rncyt-Hist.XStatus\_Rango\_Fe\_Co. \\
	\rowcolor{gray1}CYT\_RNCYT2\_HST\_X\_RG\_FCH\_EVAL &   Rncyt - Hist. X Rango\_Fech\_Eva. \\
	\rowcolor{gray1}CYT\_RNCYT2\_INST\_PROC\_CONCLUIDO &  Intituciones Proceso Concluido. \\
	\rowcolor{gray1}CYT\_RNCYT2\_PADRON\_CON\_DOM\_MAIL &  Padrón con domicilio y correos. \\
	\rowcolor{gray1}CYT\_RNCYT2\_PADRON\_VIGENTE &  Rncyt - Padrón Vigente Reing. \\
	CYT\_RNCYT2\_PADRON\_VIG\_RG\_FECH &  Padron\_vig\_rg\_fecha. \\
	\rowcolor{gray1}CYT\_RNCYT2\_PADRON\_VIG\_SOLICIT &  Rncyt - Padrón Vig. por Solic. \\
	CYT\_RNCYT2\_PADRON\_VIG\_SOLICIT2 &  Rncyt - Padrón Vig. por Solic. \\
	CYT\_RNCYT2\_PA\_CON\_DOM\_MAIL\_LET &  Pa. cn dom. y mails con letra. \\
	CYT\_RNCYT2\_REP\_CON\_FECHA\_VENC &  Reporte Fecha de Vencimiento. \\
	CYT\_RNCYT2\_REP\_CON\_FECHA\_VENC2 &  Reporte Fecha de Vencimiento 2. \\
\end{tabular}



\begin{tabular}{ m{.46\textwidth} m{.49\textwidth}  }%
	\rowcolor{gray1} {\bf Campo} &  {\bf Descripción} \\ \hline \hline

	\rowcolor{gray1}CYT\_RNCYT2\_REP\_CON\_FECHA\_VENC3 &  Reporte Fecha de Vencimiento. \\
	\rowcolor{gray1}CYT\_RNCYT2\_SOLIC\_DICT\_X\_EVALUA &  Solicit. Dictaminadas x Eval. \\
	CYT\_RNCYT2\_TOTAL\_REG & Total de registros reniecyt. \\
	\rowcolor{gray1}CYT\_RNCYT3\_CAMBIO\_RLEGAL & Reporte Cambios de R Legal. \\
	CYT\_RNCYT\_ACTUALIZA &  Última Actualización x Solic. \\
	CYT\_RNCYT\_EMPRESAS\_VIGENTES &  EMPRESAS VIGENTES. \\
	\rowcolor{gray1}CYT\_RNCYT\_ENVIO\_SOLIC & Solicitudes Enviadas. \\
	\rowcolor{gray1}CYT\_RNCYT\_ESTRUC\_JURIDICA&   Instit. y Centros Estruc. Jur.  \\
	\rowcolor{gray1}CYT\_RNCYT\_FRM & Solicitudes\_Firma\_Electronica. \\
	CYT\_RNCYT\_MOV\_USUARIO & Movimientos\_X\_usuario. \\
	CYT\_RNCYT\_PADRON\_NOVIGENTE   &    Padrón Registros NO Vigentes. \\
	CYT\_RNCYT\_PROY\_INV   & Reniecyt\_Proyectos\_Investigac. \\
	CYT\_RNCYT\_REG\_NO\_CONCLUIDOS  &Registros no concluidos. \\
	\rowcolor{gray1}CYT\_RNCYT\_REG\_VIGENTES\_RFC & Registros Vigentes con RFC. \\
	 CYT\_RNCYT\_REG\_VIGEN\_Y\_NO\_VIGEN  & Registros Rncyt Vig y No Vigen. \\
	\rowcolor{gray1}CYT\_RNCYT\_REVISION\_SOLIC &Reporte\_Revisión\_Solicitudes. \\
	\rowcolor{gray1}CYT\_RNCYT\_SECTOR\_RAMA\_CLASE   &Reg. Emp. Sector Rama y Clase. \\
	CYT\_RNCYT\_SOLICITUDES   &Relación de Solicitudes. \\
	\rowcolor{gray1}CYT\_RNCYT\_SOLIC\_DIAS\_ATRASO  &Solicitudes\_dias\_de\_atraso. \\
	CYT\_RNCYT\_SOL\_DICTAMINADAS& Solicitudes Rncyt Dictaminadas. \\
	\rowcolor{gray1}CYT\_RNCYT\_TAMAÑO\_EMPRESA &  Registros Rncyt Tamaño Empresa. \\
	 CYT\_RNCYT\_TIEMPOS\_SESION&  Tiempos Establ x Solic\_Sesion. \\
	 CYT\_SOLICITUDES\_A\_RENIECYT&Solicitu q ha recibido una emp.  \\
\end{tabular}


%=========================================================
\section{Reportes generados por PeopleSoft}
\label{appendix:Reportes:PeopleSoft}
%---------------------------------------------------------
\subsection{Apoyos que ha recibido una emp}
\label{appendix:Reportes:PeopleSoft:ApoyosRecibidosEmpresa}

Este reporte muestra los apoyos de CONACYT que ha recibido una empresa.\\

\begin{tabular}{ m{.26\textwidth} m{.61\textwidth}  }%
	\rowcolor{gray1} {\bf Campo} &  {\bf Descripción} \\ \hline \hline

	Reniecyt & Es el número de registro en Reniecyt.\\
	\rowcolor{gray1}Fondo & Se refiere al número de fondo. Identifica el área que tiene ese fondo.\\
	Descripción & Es el nombre del fondo.\\
	\rowcolor{gray1}Proyecto & Se refiere al número con el que se identifica el proyecto.\\
	Título & Se refiere al nombre del proyecto.\\
	\rowcolor{gray1}Convocatoria & Es la clave de la convocatoria.\\
\end{tabular}

\newpage
%%%%%%%%%%%%%%%%%%%%%%%%%%%%%%%%%%%%%%%%%%%%%%%%%%%%%%%%%%%%%%%%%%%%%
\subsection{Solicitudes en Convocatoria Vigente}
\label{appendix:Reportes:PeopleSoft:ConvocatoriaVigente}

Este reporte muestra las solicitudes registradas para la Convocatoria Vigente.\\

\begin{tabular}{ m{.26\textwidth} m{.61\textwidth}  }%
	\rowcolor{gray1} {\bf Campo} &  {\bf Descripción} \\ \hline \hline

	Reg. CVU.  & Significa el número de CVU del \refActor{Usuario}.\\
	\rowcolor{gray1}Solicitud. & Nombre de la Persona Física con Perfil Académico.\\
	Datos adicionales.  & Datos adicionales registrados en CVU.\\
	\rowcolor{gray1}Fecha de estatus.& Es igual a fecha de recepción.\\
	 Estatus.& En este campo pueden existir seis tipos, los cuales pueden ser {\bf solicitud recibida, falta de información,  cancelada, inscrita, enviada a evaluar y notificación de aceptado}.\\
	
\end{tabular}

\newpage
%---------------------------------------------------------
\subsection{Reporte Cambios de R Legal}
\label{appendix:Reportes:PeopleSoft:ReporteCambiosRLegal}

Cuando se realiza algún cambio en el \refActor{Representante Legal} de la entidad, se genera el siguiente reporte por nombre de entidad. Si no existen cambios, las únicas columnas con información serán las que conteniendo la palabra ``actual''.\\

\begin{tabular}{ m{.26\textwidth} m{.61\textwidth}  }%
	\rowcolor{gray1} {\bf Campo} &  {\bf Descripción} \\ \hline \hline

	Id\_Solicitante & Es el número de folio o el número de solicitud.\\
	\rowcolor{gray1}Sede & Se refiere al número de folio o de solicitud, antecedido por la clave de la sede.\\
	Centro & Es el número identificador de la subsede, puede tenerlo o no.\\
	\rowcolor{gray1}Nombre empresa(anterior) & Es el nombre que tenía la Institución.\\
	Nombre empresa (actual) & Es el nuevo nombre de la Institución. Puede haber tenido o no algún cambio.\\
	\rowcolor{gray1}Representante legal (anterior) & Es el nombre del antiguo Representante Legal de la Institución.\\
	Representante legal (actual) & Es el nombre del actual Representante Legal de la Institución.\\
	\rowcolor{gray1}RFC (anterior) & Se refiere a la clave del Registro Federal de Contribuyentes anterior de la entidad.\\
	RFC (actual) & Se refiere a la clave del Registro Federal de Contribuyentes actual de la entidad.\\
	\rowcolor{gray1}Fecha cambio (anterior) & Se refiere a la fecha de cuando se tuvo un primer cambio.\\
	Fecha cambio (actual) & Se refiere a la fecha de cuando se tuvo un segundo cambio.\\
	\rowcolor{gray1}Número de registro (anterior) & Se refiere al número de registro Reniecyt anterior.\\
	Número de registro(actual) & Se refiere al número de registro Reniecyt actual.\\
	\rowcolor{gray1}Fecha vigencia (anterior) & Se refiere a la fecha de vigencia del número de registro Reniecyt anterior.\\
	Fecha vigencia (actual) & Se refiere a la fecha de vigencia del número de registro Reniecyt actual.\\
	\rowcolor{gray1}Usuario que realizo el cambio & Se refiere al nombre del Representante Legal, Titular o Enlace que realizo el cambio.\\
	Fecha y hora & Se refiere a la Fecha y a la Hora en que se realizo el cambio.\\
\end{tabular}
\newpage
%---------------------------------------------------------
\subsection{Rncyt-Historial por Solicit.}
\label{appendix:Reportes:PeopleSoft:RncytHistorialSolicitud}

El siguiente reporte refleja toda la historia de la solicitud (ingreso de información, revisión, corrección, coordinador, evaluador, etc.) a lo largo del proceso, hasta obtener un dictamen definitivo.\\

\begin{tabular}{ m{.26\textwidth} m{.61\textwidth}  }%
	\rowcolor{gray1} {\bf Campo} &  {\bf Descripción} \\ \hline \hline

	Clave de Entidad & Es la clave de la institución que le asigna el catálogo de Entidades.\\
	\rowcolor{gray1}Tipo\_Solicitud & Es el tipo de institución o empresa al que pertenece la solicitud, pueden ser: personas físicas con actividad empresarial (PFE), empresas (CEM), instituciones privadas no lucrativas (IPL), instituciones de enseñanza superior - sedes (IES), instituciones de enseñanza superior - subsedes (IES), centros de investigación - sedes (CI), centros de investigación - subsedes(CI), instituciones de la administración pública - sedes (IDAP), instituciones de la administración pública - subsedes (IDAP).\\
	id\_Solicitante & Es el número de folio o el número de solicitud.\\
	\rowcolor{gray1}Sede & Se refiere al número de folio o de solicitud, antecedido por la clave de la sede.\\
	Centro & Es el número identificador de la subsede, puede tenerlo o no.\\
	\rowcolor{gray1}Nombre Emp./Inst./Pers. Fís. & Es el nombre de la institución/Entidad o persona física con actividad empresarial.\\
	Estatus & Es el estatus que aparece de acuerdo a su proceso, el cual puede ser:
               \begin{description}
                    \item[P] proceso de captura.
                    \item[1] recibido.
                    \item[4] falta de información.
                    \item[2] coordinador.
                    \item[R] reasignada.
                    \item[A] evaluador (la solicitud fue asignada en tiempo).
                    \item[S] evaluador, no evalúa y se le reasigna, se va a sesión por que fue dictaminada de Mayor Información.
                    \item[7] falta de información y anexos incorrectos.
                    \item[3] aceptado por el sistema.
                    \item[B] mas información a consideración del evaluador.
                    \item[C] concluido el proceso total.
                    \item[5] anexos incorrectos.
               \end{description} \\
\end{tabular}

\begin{tabular}{ m{.26\textwidth} m{.61\textwidth}  }%
	\rowcolor{gray1}Descr & Se refiere a la descripción del estatus.\\
	Fecha de Inicio & Se refiere a la fecha y a la hora en el que el proceso se encuentra.\\
	\rowcolor{gray1}Fecha\_Fin\_Asignación & Se refiere a la fecha final de cuando se termina la actividad conforme al proceso. \\
	id\_Área & Es el número con el que se se identifican las áreas de asignación.\\
	\rowcolor{gray1}Descrip.Área & Es el nombre del área de asignación.\\
	\rowcolor{gray1} {\bf Campo} &  {\bf Descripción} \\ \hline \hline
	CVU\_Coord & Es el número de CVU del coordinador de área de asignación (dirección adjunta).\\
	\rowcolor{gray1}CVU\_Eval & Es el número de CVU del evaluador.\\
	Nombre\_Evaluador & Se refiere al nombre del evaluador.\\
	\rowcolor{gray1}id\_Sesión & Se refiere al identificador asignado a la sesión, solo cuando se reasigna sesión.\\
	Fecha\_Sesion & Se refiere a la fecha de la sesión.\\
	\rowcolor{gray1}Usuario\_Ultimo\_Movim. & Es el identificador del usuario que realizo el último movimiento.\\
	Comentarios al Usuario ó Coord & Son todos aquellos mensajes dirigidos al usuario o al coordinador.\\
\end{tabular}

%---------------------------------------------------------
\subsection{Rncyt-Hist. X Status\_Rango\_Fec}
\label{appendix:Reportes:PeopleSoft:RncytHistStatusRangoFecha}

Este reporte muestra las solicitudes por estatus, en determinado rango de tiempo.\\

\begin{tabular}{ m{.26\textwidth} m{.61\textwidth}  }%
	\rowcolor{gray1} {\bf Campo} &  {\bf Descripción} \\ \hline \hline

	Clave de Entidad & Es la clave de la institución que le asigna el Catálogo de Entidades.\\
	\rowcolor{gray1}Id Solic & Es el número de folio o el número de solicitud.\\
	Id Sede & Se refiere al número de folio o de solicitud, antecedido por la clave de la sede.\\
	\rowcolor{gray1}Centro & Es el número identificador de la subsede, puede tenerlo o no.\\
	Tipo de Entidad & \\
	\rowcolor{gray1}Nombre de Entidad & Nombre de la entidad.\\
	Dirección & Domicilio de la entidad.\\
	\rowcolor{gray1}Nombre\_Rep\_Legal & Nombre del Representante Legal de la entidad.\\
	Correo-E & Correo electrónico del Representante Legal.\\
	\rowcolor{gray1}Comentarios\_Evaluador & Comentarios que realiza el evaluador al momento de evaluar la solicitud.\\
    \end{tabular}

\begin{tabular}{ m{.26\textwidth} m{.61\textwidth}  }%
\rowcolor{gray1} {\bf Campo} &  {\bf Descripción} \\ \hline \hline

	Status Rncyt & Es el estatus que aparece de acuerdo a su proceso, el cual puede ser:
               \begin{description}
                    \item[P] proceso de captura.
                    \item[1] recibido.
                    \item[4] falta de información.
                    \item[2] coordinador.
                    \item[R] reasignada.
                    \item[A] evaluador (la solicitud fue asignada en tiempo).
                    \item[S] evaluador, no evalúa y se le reasigna, se va a sesión por que fue dictaminada de Mayor Información.
                    \item[7] falta de información y anexos incorrectos.
                    \item[3] aceptado por el sistema.
                    \item[B] mas información a consideración del evaluador.
                    \item[C] concluido el proceso total.
                    \item[5] anexos incorrectos.
               \end{description}\\
	\rowcolor{gray1}Descripción del estatus & Nombre del estatus Reniecyt.\\
	F Solic & \\
	\rowcolor{gray1}F Final & \\
	Areas & Es el identificador el área.\\
	\rowcolor{gray1}Descr & Se refiere al nombre del área.\\
	CVU Coordinador & Es el número de CVU del coordinador de área de asignación (dirección adjunta).\\
	\rowcolor{gray1}Nombre Coordinador & Es le nombre del coordinador.\\
	Evaluador & Es el número de CVU del evaluador.\\
	\rowcolor{gray1}Nombre & Es el nombre del evaluador\\
	Sesión & Se refiere al número de la sesión.\\
	\rowcolor{gray1}Fecha & Se refiere a la fecha de la sesión.\\
	Usuario & Es el identificador del usuario que realizo el último movimiento.\\
	\rowcolor{gray1}Comentarios & Son todos aquellos mensajes dirigidos al usuario o al coordinador.\\
\end{tabular}

\newpage
%---------------------------------------------------------
\subsection{Rncyt-Hist.XStatus\_Rango\_Fe\_Co}
\label{appendix:Reportes:PeopleSoft:RncytHistStatusRangoFeCo}

Este reporte muestra las solicitudes asignadas a los coordinadores, en determinado rango de tiempo.\\

\begin{tabular}{ m{.26\textwidth} m{.61\textwidth}  }%
	\rowcolor{gray1} {\bf Campo} &  {\bf Descripción} \\ \hline \hline

	Clave de Entidad & Es la clave de la institución que le asigna el catálogo de Entidades.\\
	\rowcolor{gray1}Tipo\_Solicitud & Es el tipo de institución o empresa al que pertenece la solicitud, pueden ser: personas físicas con actividad empresarial (PFE), empresas (CEM), instituciones privadas no lucrativas (IPL), instituciones de enseñanza superior - sedes (IES), instituciones de enseñanza superior - subsedes (IES), centros de investigación - sedes (CI), centros de investigación - subsedes(CI), instituciones de la administración pública - sedes (IDAP), instituciones de la administración pública - subsedes (IDAP).\\
	id\_Solicitante & Es el número de folio o el número de solicitud.\\
	\rowcolor{gray1}Sede & Se refiere al número de folio o de solicitud, antecedido por la clave de la sede.\\
	Centro & Es el número identificador de la subsede, puede tenerlo o no.\\
	\rowcolor{gray1}Nombre Emp./Inst./Pers.Fís. & Es el nombre de la institución/Entidad o persona física con actividad empresarial.\\
	Status & Es el estatus que aparece de acuerdo a su proceso, el cual puede ser:
               \begin{description}
                    \item[P] proceso de captura.
                    \item[1] recibido.
                    \item[4] falta de información.
                    \item[2] coordinador.
                    \item[R] reasignada.
                    \item[A] evaluador (la solicitud fue asignada en tiempo).
                    \item[S] evaluador, no evalúa y se le reasigna, se va a sesión por que fue dictaminada de Mayor Información.
                    \item[7] falta de información y anexos incorrectos.
                    \item[3] aceptado por el sistema.
                    \item[B] mas información a consideración del evaluador.
                    \item[C] concluido el proceso total.
                    \item[5] anexos incorrectos.
               \end{description}\\
\end{tabular}

\begin{tabular}{ m{.26\textwidth} m{.61\textwidth}  }%
	\rowcolor{gray1} {\bf Campo} &  {\bf Descripción} \\ \hline \hline

	\rowcolor{gray1}Descrip. Status & Se refiere a la descripción del estatus.\\
	Fecha\_Fin\_Asignación & Se refiere a la fecha final de cuando se termina la actividad conforme al proceso.\\
	\rowcolor{gray1}id\_Área & Es el número con el que se se identifican las áreas de asignación.\\
	Descrip.Área & Es el nombre del área de asignación.\\
	\rowcolor{gray1}CVU\_Coord & Es el número de CVU del coordinador de área de asignación (dirección adjunta).\\
	CVU\_Eval & Es el número de CVU del evaluador.\\
	\rowcolor{gray1}Nombre\_Evaluador & Se refiere al nombre del evaluador.\\
	id\_Sesión & Se refiere al identificador asignado a la sesión, solo cuando se reasigna sesión.\\
	\rowcolor{gray1}Fecha\_Sesión & Se refiere a la fecha de la sesión.\\
	Usuario & Es el identificador del usuario que realizo el último movimiento.\\
	\rowcolor{gray1}Comentarios al Usuario ó Coord & Son todos aquellos mensajes dirigidos al usuario o al coordinador.\\
\end{tabular}

\newpage
%---------------------------------------------------------
\subsection{Rncyt-Hist. X Rango\_Eva}
\label{appendix:Reportes:PeopleSoft:RncytHistRangoEva}

Este reporte muestra las solicitudes asignadas a los dictaminadores, en un determinado rango de tiempo.\\

\begin{tabular}{ m{.26\textwidth} m{.61\textwidth}  }%
	\rowcolor{gray1} {\bf Campo} &  {\bf Descripción} \\ \hline \hline

	Clave de Entidad & Es la clave de la institución que le asigna el catálogo de Entidades.\\
	\rowcolor{gray1}Tipo\_Solicitud & Es el tipo de institución o empresa al que pertenece la solicitud, pueden ser: personas físicas con actividad empresarial (PFE), empresas (CEM), instituciones privadas no lucrativas (IPL), instituciones de enseñanza superior - sedes (IES), instituciones de enseñanza superior - subsedes (IES), centros de investigación - sedes (CI), centros de investigación - subsedes(CI), instituciones de la administración pública - sedes (IDAP), instituciones de la administración pública - subsedes (IDAP).\\
	id\_Solicitante & Es el número de folio o el número de solicitud.\\
	\rowcolor{gray1}Sede & Se refiere al número de folio o de solicitud, antecedido por la clave de la sede.\\
	Centro & Es el número identificador de la subsede, puede tenerlo o no.\\
	\rowcolor{gray1}Nombre Emp./Inst./Pers.Fís. & Es el nombre de la institución/Entidad o persona física con actividad empresarial.\\
	Status & Es el estatus que aparece de acuerdo a su proceso, el cual puede ser:
               \begin{description}
                    \item[P] proceso de captura.
                    \item[1] recibido.
                    \item[4] falta de información.
                    \item[2] coordinador.
                    \item[R] reasignada.
                    \item[A] evaluador (la solicitud fue asignada en tiempo).
                    \item[S] evaluador, no evalúa y se le reasigna, se va a sesión por que fue dictaminada de Mayor Información.
                    \item[7] falta de información y anexos incorrectos.
                    \item[3] aceptado por el sistema.
                    \item[B] mas información a consideración del evaluador.
                    \item[C] concluido el proceso total.
                    \item[5] anexos incorrectos.
               \end{description}\\
\end{tabular}

\begin{tabular}{ m{.26\textwidth} m{.61\textwidth}  }%
	\rowcolor{gray1} {\bf Campo} &  {\bf Descripción} \\ \hline \hline
	\rowcolor{gray1}Descrip. Status & Se refiere a la descripción del estatus.\\
	Fecha y Hora & e refiere a la fecha y a la hora en el que el proceso se encuentra.\\
	\rowcolor{gray1}Fecha\_Fin\_Asignación &Se refiere a la fecha final de cuando se termina la actividad conforme al proceso.\\
	id\_Área & Es el número con el que se se identifican las áreas de asignación.\\
	\rowcolor{gray1}Descrip.Área & Es el nombre del área de asignación.\\
	CVU\_Coord & Es el número de CVU del coordinador de área de asignación (dirección adjunta).\\
	\rowcolor{gray1}CVU\_Eval & Es el número de CVU del evaluador.\\
	Nombre\_Evaluador & Se refiere al nombre del evaluador.\\
	\rowcolor{gray1}id\_Sesión & Se refiere al identificador asignado a la sesión, solo cuando se reasigna sesión.\\
	Fecha\_Sesión & Se refiere a la fecha de la sesión.\\
	\rowcolor{gray1}Usuario & Es el identificador del usuario que realizo el último movimiento.\\
	Comentarios al Usuario ó Coord & Son todos aquellos mensajes dirigidos al usuario o al coordinador.\\
\end{tabular}

\newpage
%---------------------------------------------------------
\subsection{Instituciones Proceso Concluido}
\label{appendix:Reportes:PeopleSoft:InstitucionesProcesoConcluido}

Este reporte muestra las solicitudes que han concluido el proceso RENIECYT.\\

\begin{tabular}{ m{.26\textwidth} m{.61\textwidth}  }%
	\rowcolor{gray1} {\bf Campo} &  {\bf Descripción} \\ \hline \hline

	Clave de Entidad & Es la clave de la institución que le asigna el catálogo de Entidades.\\
	\rowcolor{gray1}Des. Campo & Nombre de la entidad.\\
	Nombre & Nombre del representante legal.\\
	\rowcolor{gray1}Descr Larga & Domicilio de la entidad.\\
\end{tabular}

%---------------------------------------------------------
\subsection{Padrón con domicilio y correos}
\label{appendix:Reportes:PeopleSoft:PadronDomicilioyCorreos}

Este reporte muestra los datos de la entidad, Representante Legal, Títular y Enlace de las solicitudes.\\

\begin{tabular}{ m{.26\textwidth} m{.61\textwidth}  }%
	\rowcolor{gray1} {\bf Campo} &  {\bf Descripción} \\ \hline \hline

	Clave de Entidad & Es la clave de la institución que le asigna el catálogo de Entidades.\\
	\rowcolor{gray1}Tipo de Solicitud & Es el tipo de institución o empresa al que pertenece la solicitud, pueden ser: personas físicas con actividad empresarial (PFE), empresas (CEM), instituciones privadas no lucrativas (IPL), instituciones de enseñanza superior - sedes (IES), instituciones de enseñanza superior - subsedes (IES), centros de investigación - sedes (CI), centros de investigación - subsedes(CI), instituciones de la administración pública - sedes (IDAP), instituciones de la administración pública - subsedes (IDAP).\\
	Id\_Solicitante & Es el número de folio o el número de solicitud.\\
	\rowcolor{gray1}Sede & Se refiere al número de folio o de solicitud, antecedido por la clave de la sede.\\
	Centro & Es el número identificador de la subsede, puede tenerlo o no.\\
	\rowcolor{gray1}Nombre Empresa/Institución & Es el nombre de la institución/entidad.\\
	Entidad Federativa & Nombre del Estado de la República al que pertenece la entidad.\\
	\rowcolor{gray1}Domicilio & Domicilio de la entidad.\\
	Nombre\_Legal & Nombre del representante legal de la entidad.\\
	\rowcolor{gray1}Correo Electrónico-Legal & Correo electrónico del representante legal de la entidad.\\
	Nombre\_Titular & Nombre del titular de la entidad.\\
	\rowcolor{gray1}Correo Electrónico-Titular & Correo electrónico del titular de la entidad.\\
	Nombre\_Enlace & Nombre del enlace de la entidad.\\
	\rowcolor{gray1}Correo Electrónico-Enlace & Correo electrónico del enlace de la entidad.\\
	Teléfono Enlace & Es el número telefónico del enlace.\\
\end{tabular}

\newpage
%---------------------------------------------------------
\subsection{Rncyt-Padrón Vigente Reing.}
\label{appendix:Reportes:PeopleSoft:RncytPadronVigenteReing}

Este reporte muestra las solicitudes dependiendo de la dirección regional a la que pertenecen.\\

\begin{tabular}{ m{.26\textwidth} m{.61\textwidth}  }%
	\rowcolor{gray1} {\bf Campo} &  {\bf Descripción} \\ \hline \hline

	Región & Dirección regional a la que pertenece la entidad.\\
	\rowcolor{gray1}Estado & Nombre del Estado de la República al que pertenece la entidad.\\
	Tipo de Solicitud & Es el tipo de institución o empresa al que pertenece la solicitud, pueden ser: personas físicas con actividad empresarial (PFE), empresas (CEM), instituciones privadas no lucrativas (IPL), instituciones de enseñanza superior - sedes (IES), instituciones de enseñanza superior - subsedes (IES), centros de investigación - sedes (CI), centros de investigación - subsedes(CI), instituciones de la administración pública - sedes (IDAP), instituciones de la administración pública - subsedes (IDAP).\\
	\rowcolor{gray1}Total & Número total por cada tipo de solicitud.\\
\end{tabular}


%---------------------------------------------------------
\subsection{Rncyt - Padrón Vig. por Solic}
\label{appendix:Reportes:PeopleSoft:PadronVigSol}

Este reporte muestra la fecha de inicio y de vigencia del registro de las solicitudes.\\

\begin{tabular}{ m{.26\textwidth} m{.61\textwidth}  }%
	\rowcolor{gray1} {\bf Campo} &  {\bf Descripción} \\ \hline \hline

	Clave de Entidad & Es la clave de la institución que le asigna el catálogo de Entidades.\\
	\rowcolor{gray1}Tipo de Solicitud & Es el tipo de institución o empresa al que pertenece la solicitud, pueden ser: personas físicas con actividad empresarial (PFE), empresas (CEM), instituciones privadas no lucrativas (IPL), instituciones de enseñanza superior - sedes (IES), instituciones de enseñanza superior - subsedes (IES), centros de investigación - sedes (CI), centros de investigación - subsedes(CI), instituciones de la administración pública - sedes (IDAP), instituciones de la administración pública - subsedes (IDAP).\\
	Id Solic & Es el número de folio o el número de solicitud.\\
	\rowcolor{gray1}Sede & Se refiere al número de folio o de solicitud, antecedido por la clave de la sede.\\
	Centro & Es el número identificador de la subsede, puede tenerlo o no.\\
	\rowcolor{gray1}Nombre Empresa/Institución & Es el nombre de la entidad.\\
	País & Nombre del país al que pertenece la entidad\\
	\rowcolor{gray1}Estado & Nombre del Estado de la República al que pertenece la entidad.\\
	id\_Registro & Es el número identificador del registro de la entidad.\\
	\rowcolor{gray1}Fecha\_Registro & Es la fecha en la que se registro la entidad.\\
	Fecha\_Ultima\_Actualizac. & Es la fecha en la que se llevo a cabo el último movimiento en los datos de la entidad.\\
	\rowcolor{gray1}Fecha\_Fin\_Registro & Es la fecha de vencimiento de registro.\\
\end{tabular}

%---------------------------------------------------------
\subsection{Reporte Fecha de Vencimiento}
\label{appendix:Reportes:PeopleSoft:}

Este reporte muestra la fecha de vigencia del registro de las solicitudes.\\

\begin{tabular}{ m{.26\textwidth} m{.61\textwidth}  }%
	\rowcolor{gray1} {\bf Campo} &  {\bf Descripción} \\ \hline \hline

	Clave de Entidad & Es la clave de la institución que le asigna el catálogo de Entidades.\\
	\rowcolor{gray1}Tipo de Solicitud & Es el tipo de institución o empresa al que pertenece la solicitud, pueden ser: personas físicas con actividad empresarial (PFE), empresas (CEM), instituciones privadas no lucrativas (IPL), instituciones de enseñanza superior - sedes (IES), instituciones de enseñanza superior - subsedes (IES), centros de investigación - sedes (CI), centros de investigación - subsedes(CI), instituciones de la administración pública - sedes (IDAP), instituciones de la administración pública - subsedes (IDAP).\\
	Id\_Solicitante & Es el número de folio o el número de solicitud.\\
	\rowcolor{gray1}Sede & Se refiere al número de folio o de solicitud, antecedido por la clave de la sede.\\
	Centro & Es el número identificador de la subsede, puede tenerlo o no.\\
	\rowcolor{gray1}Nombre Empresa/Institución & Nombre de la entidad.\\
	Registro & Es el número de registro asignado a la entidad.\\
	\rowcolor{gray1}Fecha\_Fin\_Registro & Es la fecha de vigencia del registro.\\
	Dirección & Dirección de la entidad.\\
	\rowcolor{gray1}Representante Legal & Nombre del representante legal de la entidad.\\
	Emal Leg & Correo electrónico del representante legal de la entidad.\\
	\rowcolor{gray1}Est & Nombre del Estado de la República al que pertenece la entidad.\\
\end{tabular}

%---------------------------------------------------------
\subsection{Solicit. Dictaminadas x Eval.}
\label{appendix:Reportes:PeopleSoft:SolicitDictaminadaxEval}

Este reporte muestra el número de solicitudes que han evaluado cada uno de los dictaminadores.\\

\begin{tabular}{ m{.26\textwidth} m{.61\textwidth}  }%
	\rowcolor{gray1} {\bf Campo} &  {\bf Descripción} \\ \hline \hline

	Id\_Evaluador & Se refiere al número identificador del evaluador.\\
	\rowcolor{gray1}Nombre Del Evaluador & Es el nombre del evaluador.\\
	Dictamen & Se refiere al estatus que puede llegar a tener un dictamen, pueden ser: Aceptadas, Negadas o de Mas información.\\
	\rowcolor{gray1}Total\_x\_Dictamen & Es el número total de dictámenes evaluados.\\
\end{tabular}

%---------------------------------------------------------
\subsection{Reporte Cambios de R Legal}
\label{appendix:Reportes:PeopleSoft:ReporteRLegal}

Cuando se realiza algún cambio en el \refActor{Representante Legal} de la entidad, se genera el siguiente reporte por día. Si no existen cambios, las unicas columnas con información serán las que conteniendo la palabra ``actual''.\\


\begin{tabular}{ m{.26\textwidth} m{.61\textwidth}  }%
	\rowcolor{gray1} {\bf Campo} &  {\bf Descripción} \\ \hline \hline

	Id\_Solicitante & Es el número de folio o el número de solicitud.\\
	\rowcolor{gray1}Centro & Es el número identificador de la subsede, puede tenerlo o no.\\
	Sede & Se refiere al número de folio o de solicitud, antecedido por la clave de la sede.\\
	\rowcolor{gray1}Nombre Empresa (Anterior) & Es el nombre de la empresa anterior.\\
	Nombre Empresa (Actual) & Es el nombre de la empresa  actual.\\
	\rowcolor{gray1}Representante Legal Anterior  & Nombre del representante legal anterior.\\
	Representante Legal Actual & Nombre del representante legal actual.\\
	\rowcolor{gray1}RFC Anterior & Se refiere a la clave del Registro Federal de Contribuyentes anterior de la entidad.\\
	RFC Actual & Se refiere a la clave del Registro Federal de Contribuyentes actual de la entidad.\\
	\rowcolor{gray1}Fecha Cambio Anterior & Se refiere a la fecha de cuando se tuvo un primer cambio.\\
	Fecha Cambio Actual & Se refiere a la fecha de cuando se tuvo un segundo cambio.\\
	\rowcolor{gray1}Número de Registro Anterior & Se refiere al número de registro Reniecyt anterior.\\
	Número de Registro Actual & Se refiere al número de registro Reniecyt actual.\\
	\rowcolor{gray1}Fecha Vigencia Anterior & Se refiere a la fecha de vigencia del registro Reniecyt anterior.\\
	Fecha Vigencia Actual & Se refiere a la fecha de vigencia del registro Reniecyt actual.\\
\end{tabular}


%---------------------------------------------------------
\subsection{Solicitudes enviadas}
\label{appendix:Reportes:PeopleSoft:SolicitudesEnviadas}
%Este reporte se genera cuando las solicitudes arriban al sistema de peopleSoft

Este reporte muestra las solicitudes que han llegado al módulo de administración del RENIECYT para ser revisadas.\\%El siguiente reporte puede ser consultado por fecha, ID o nombre de la solicitud.

\begin{tabular}{ m{.26\textwidth} m{.61\textwidth}  }%
	\rowcolor{gray1} {\bf Campo} &  {\bf Descripción} \\ \hline \hline

	Tipo solicitud & Es el tipo de institución o empresa al que pertenece la solicitud, pueden ser: personas físicas con actividad empresarial (PFE), empresas (CEM), instituciones privadas no lucrativas (IPL), instituciones de enseñanza superior - sedes (IES), instituciones de enseñanza superior - subsedes (IES), centros de investigación - sedes (CI), centros de investigación - subsedes(CI), instituciones de la administración pública - sedes (IDAP), instituciones de la administración pública - subsedes (IDAP).\\
	\rowcolor{gray1}Solicitud/CVU & Es el número de folio.\\
	Sede & Se refiere al número de folio o de solicitud, antecedido por la clave de la sede.\\
	\rowcolor{gray1}Identificador Centro & Es el número identificador de la subsede, puede tenerlo o no.\\
	Nombre de la solicitud & Es el nombre de la solicitud.\\
	\rowcolor{gray1}Fecha y Hora & Es la fecha y hora en que arribó la solicitud a RENIECYT.\\
    Status Solicitud & Se refiere al estatus que presenta la solicitud.\\
	\rowcolor{gray1}Entidad Federativa & Nombre del Estado de la República al que pertenece la entidad.\\
	Tipo Empresa & Se refiere al tamaño de la empresa.\\
\end{tabular}


%---------------------------------------------------------
\subsection{Instit. y Centros Estruc. Jur.}
\label{appendix:Reportes:PeopleSoft:InstyCentrosEstrucJur}
%Este reporte se genera cuando las solicitudes arriban al sistema de peopleSoft

Este reporte muestra la estructura jurídica de las entidades.\\

\begin{tabular}{ m{.26\textwidth} m{.61\textwidth}  }%
	\rowcolor{gray1} {\bf Campo} &  {\bf Descripción} \\ \hline \hline

	Id Solicitud & Es el tipo de institución o empresa al que pertenece la solicitud, pueden ser: personas físicas con actividad empresarial (PFE), empresas (CEM), instituciones privadas no lucrativas (IPL), instituciones de enseñanza superior - sedes (IES), instituciones de enseñanza superior - subsedes (IES), centros de investigación - sedes (CI), centros de investigación - subsedes(CI), instituciones de la administración pública - sedes (IDAP), instituciones de la administración pública - subsedes (IDAP).\\
	\rowcolor{gray1}Id Sede & Se refiere al número de folio o de solicitud, antecedido por la clave de la sede.\\
	Nombre Institución/Centro & Es el número identificador de la subsede, puede tenerlo o no.\\
	\rowcolor{gray1}Estructura Jurídica & Es la clasificación de la entidades acorde a su estructura jurídica.\\
	Estado & Nombre del Estado de la República al que pertenece la entidad.\\
	\rowcolor{gray1}Registro & Es el número de registro Reniecyt.\\
	Fecha Registro & Es la fecha en que se registro.\\
	\rowcolor{gray1}Fecha Vigencia & Fecha de vencimiento del registro.\\
\end{tabular}

\newpage
%---------------------------------------------------------
\subsection{Solicitudes\_Firma\_Electrónica}
\label{appendix:Reportes:PeopleSoft:SolicitudesFirmaElectronicaGenerada}
%Este reporte se genera cuando se genera la firma electrónica
%El siguiente reporte puede ser consultado por fecha, ID o nombre de la solicitud.
Este reporte muestra las solicitudes que han firmado la carta de confidencialidad.\\

\begin{tabular}{ m{.26\textwidth} m{.61\textwidth}  }%
	\rowcolor{gray1} {\bf Campo} &  {\bf Descripción} \\ \hline \hline

	Tipo de Solicitud & Es el tipo de institución o empresa al que pertenece la solicitud, pueden ser: personas físicas con actividad empresarial (PFE), empresas (CEM), instituciones privadas no lucrativas (IPL), instituciones de enseñanza superior - sedes (IES), instituciones de enseñanza superior - subsedes (IES), centros de investigación - sedes (CI), centros de investigación - subsedes(CI), instituciones de la administración pública - sedes (IDAP), instituciones de la administración pública - subsedes (IDAP).\\
	\rowcolor{gray1}Solicitud/CVU & Es el número de folio.\\
	Identificador Sede & Se refiere al número de folio o de solicitud, antecedido por la clave de la sede.\\
	\rowcolor{gray1}Identificador Centro & Es el número identificador de la subsede, puede tenerlo o no.\\
	Nombre de la Solicitud & Se refiere al nombre de la solicitud.\\
	\rowcolor{gray1}CVU\_Rep\_Legal & Es el número de CVU del representante legal.\\
	Nombre\_Rep\_Legal & Es el nombre del representante legal.\\
	\rowcolor{gray1}Fecha\_Generación\_Firma & Es la fecha en la que se firma la carta de confidencialidad.\\
    Status\_Solicitud & Es el estatus que aparece de acuerdo a su proceso, el cual puede ser:
               \begin{description}
                    \item[P] proceso de captura.
                    \item[1] recibido.
                    \item[4] falta de información.
                    \item[2] coordinador.
                    \item[R] reasignada.
                    \item[A] evaluador (la solicitud fue asignada en tiempo).
                    \item[S] evaluador, no evalúa y se le reasigna, se va a sesión por que fue dictaminada de Mayor Información.
                    \item[7] falta de información y anexos incorrectos.
                    \item[3] aceptado por el sistema.
                    \item[B] mas información a consideración del evaluador.
                    \item[C] concluido el proceso total.
                    \item[5] anexos incorrectos.
               \end{description}\\
\end{tabular}

\newpage
%---------------------------------------------------------
\subsection{Reporte\_Revisión\_Solicitudes}
\label{appendix:Reportes:PeopleSoft:ReporteRevSolic}
%Este reporte se genera cuando las solicitudes arriban al sistema de peopleSoft

Este reporte muestra el estado de la revisión de las solicitudes.\\

\begin{tabular}{ m{.26\textwidth} m{.61\textwidth}  }%
	\rowcolor{gray1} {\bf Campo} &  {\bf Descripción} \\ \hline \hline

	Tipo de solicitud & Es el tipo de institución o empresa al que pertenece la solicitud, pueden ser: personas físicas con actividad empresarial (PFE), empresas (CEM), instituciones privadas no lucrativas (IPL), instituciones de enseñanza superior - sedes (IES), instituciones de enseñanza superior - subsedes (IES), centros de investigación - sedes (CI), centros de investigación - subsedes(CI), instituciones de la administración pública - sedes (IDAP), instituciones de la administración pública - subsedes (IDAP).\\
	\rowcolor{gray1}Solicitud CVU & Es el número de folio.\\
	Identificador Sede & Se refiere al número de folio o de solicitud, antecedido por la clave de la sede.\\
	\rowcolor{gray1}Identificador Centro & Es el número identificador de la subsede, puede tenerlo o no.\\
	Nombre de la solicitud & Es el nombre del entidad.\\
	\rowcolor{gray1}Entidad Federativa & Nombre del Estado de la República al que pertenece la entidad.\\
	Estruc. Juríd.-IES,CPI,IDAP,IPL & Es la clasificación de la entidades acorde a su estructura jurídica.\\
	\rowcolor{gray1}Especificar (IES,CPI,IDAP,IPL) & Es la clasificación de la entidades acorde a su estructura jurídica.\\
	Tamaño Empresa & Se refiere al tamaño de la empresa.\\
	\rowcolor{gray1}Correo Enlace Rncyt & Correo electrónico del enlace de la entidad.\\
	Status\_Rncyt & Es el estatus que aparece de acuerdo a su proceso, el cual puede ser:\\
        & \\
               & \begin{UClist}
                    \UCli[P] proceso de captura.
                    \UCli[1] recibido.
                    \UCli[4] falta de información.
                    \UCli[2] coordinador.
                    \UCli[R] reasignada.
                    \UCli[A] evaluador (la solicitud fue asignada en tiempo).
                    \UCli[S] evaluador, no evalúa y se le reasigna, se va a sesión por que fue dictaminada de Mayor Información.
                    \UCli[7] falta de información y anexos incorrectos.
                    \UCli[3] aceptado por el sistema.
                    \UCli[B] mas información a consideración del evaluador.
                    \UCli[C] concluido el proceso total.
                    \UCli[5] anexos incorrectos.
               \end{UClist}\\
	\rowcolor{gray1}Comentarios & Son todos aquellos mensajes dirigidos al usuario o al coordinador.\\
	Fecha\_Status & Se refiere a la fecha en que se realizo el último cambio de estatus.\\
\end{tabular}

\newpage
%---------------------------------------------------------
\subsection{Registros Vigentes con RFC}
\label{appendix:Reportes:PeopleSoft:RegistrosVigentesRFC}
%Este reporte se genera cuando las solicitudes arriban al sistema de peopleSoft

Este reporte muestra las solicitudes con registros vigentes con RFC.\\

\begin{tabular}{ m{.26\textwidth} m{.61\textwidth}  }%
	\rowcolor{gray1} {\bf Campo} &  {\bf Descripción} \\ \hline \hline

	Id Solicitante & Es el número de folio o el número de solicitud.\\
	\rowcolor{gray1}Id Sede & Se refiere al número de folio o de solicitud, antecedido por la clave de la sede.\\
	Id Centro & Es el número identificador de la subsede, puede tenerlo o no.\\
	\rowcolor{gray1}Nombre Empresa/Institución & Se refiere al nombre de la entidad.\\
	RFC & Es el RFC de la entidad.\\
	\rowcolor{gray1}Fecha de Registro & Es la fecha en que se le otorgo el registro Reniecyt.\\
	Num. Registro Rncyt & Es el número de registro de Reniecyt.\\
	\rowcolor{gray1}Dictamen & Son todos los dictámenes aceptados.\\
	Status & Es el estatus que aparece de acuerdo a su proceso, el cual puede ser:
               \begin{description}
                    \item[P] proceso de captura.
                    \item[1] recibido.
                    \item[4] falta de información.
                    \item[2] coordinador.
                    \item[R] reasignada.
                    \item[A] evaluador (la solicitud fue asignada en tiempo).
                    \item[S] evaluador, no evalúa y se le reasigna, se va a sesión por que fue dictaminada de Mayor Información.
                    \item[7] falta de información y anexos incorrectos.
                    \item[3] aceptado por el sistema.
                    \item[B] mas información a consideración del evaluador.
                    \item[C] concluido el proceso total.
                    \item[5] anexos incorrectos.
               \end{description}.\\
	\rowcolor{gray1}Fecha Vigencia & Es la fecha en que expira el registro.\\
\end{tabular}
\newpage
%---------------------------------------------------------
\subsection{Reg. Emp. Sector Rama y Clase}
\label{appendix:Reportes:PeopleSoft:RegEmpSectorRamaClase}
%Este reporte se genera cuando las solicitudes arriban al sistema de peopleSoft

Este reporte muestra el sector económico de las entidades.\\

\begin{tabular}{ m{.26\textwidth} m{.61\textwidth}  }%
	\rowcolor{gray1} {\bf Campo} &  {\bf Descripción} \\ \hline \hline

	ID Solic & Es el número de folio o el número de solicitud.\\
	\rowcolor{gray1}Id Sede & Se refiere al número de folio o de solicitud, antecedido por la clave de la sede.\\
	Nombre Empresa & Es el nombre de la entidad.\\
	\rowcolor{gray1}RFC & Se refiere al RFC de la entidad.\\
	Tamaño Empresa& Es de acuerdo a su actividad económica y el número de personas que laboran en la entidad.\\
	\rowcolor{gray1}Estado & Es el nombre del estado al que pertenece la entidad.\\
	Total Personas & Es el número total de personas que laboran en la entidad.\\
	\rowcolor{gray1}Sector & Se refiere al sector económico.\\
	Rama & Es la rama a la que pertenece la entidad de acuerdo al sector al que pertenezca.\\
	\rowcolor{gray1}Clase & Es la clase a la que pertenece al entidad de acuerdo a la rama a la que pertenezca.\\
	Registro & Es el número de registro.\\
	\rowcolor{gray1}Fecha Registro & Es la fecha en la que se llevo a cabo el registro de la entidad.\\
	Fecha Fin Registro & Es la fecha de vencimiento del registro de la entidad.\\
\end{tabular}

\newpage
%---------------------------------------------------------
\subsection{Solicitudes\_dias\_de\_atraso}
\label{appendix:Reportes:PeopleSoft:SolicitudesDiasAtraso}
%Este reporte se genera cuando las solicitudes arriban al sistema de peopleSoft

Este reporte muestra los días de atraso que llevan las solicitudes sin ser revisadas, en el módulo de administración.\\

\begin{tabular}{ m{.26\textwidth} m{.61\textwidth}  }%
	\rowcolor{gray1} {\bf Campo} &  {\bf Descripción} \\ \hline \hline

	Tipo de Solicitud & Es el tipo de institución o empresa al que pertenece la solicitud, pueden ser: personas físicas con actividad empresarial (PFE), empresas (CEM), instituciones privadas no lucrativas (IPL), instituciones de enseñanza superior - sedes (IES), instituciones de enseñanza superior - subsedes (IES), centros de investigación - sedes (CI), centros de investigación - subsedes(CI), instituciones de la administración pública - sedes (IDAP), instituciones de la administración pública - subsedes (IDAP).\\
	\rowcolor{gray1}Solicitud/CVU & Es el número de folio.\\
	Identificador Sede & Se refiere al número de folio o de solicitud, antecedido por la clave de la sede.\\
	\rowcolor{gray1}Identificador Centro & Es el número identificador de la subsede, puede tenerlo o no.\\
	Nombre de la Solicitud & Es el nombre del entidad.\\
	\rowcolor{gray1}Fecha de Envío & Es la fecha en que se envía la solicitud por parte de la entidad.\\
	Status\_Solicitud & Es el estatus que aparece de acuerdo a su proceso, el cual puede ser:
               \begin{description}
                    \item[P] proceso de captura.
                    \item[1] recibido.
                    \item[4] falta de información.
                    \item[2] coordinador.
                    \item[R] reasignada.
                    \item[A] evaluador (la solicitud fue asignada en tiempo).
                    \item[S] evaluador, no evalúa y se le reasigna, se va a sesión por que fue dictaminada de Mayor Información.
                    \item[7] falta de información y anexos incorrectos.
                    \item[3] aceptado por el sistema.
                    \item[B] mas información a consideración del evaluador.
                    \item[C] concluido el proceso total.
                    \item[5] anexos incorrectos.
               \end{description}\\
	\rowcolor{gray1}Días de retraso & Es el número de días que lleva de retraso la solicitud de la entidad a su revisión.\\
\end{tabular}

%---------------------------------------------------------
\subsection{Registros Rncyt Tamaño Empresa}
\label{appendix:Reportes:PeopleSoft:RegistrosRncytTamanoEmpresa}
%Este reporte se genera cuando las solicitudes arriban al sistema de peopleSoft

Este reporte muestra las solicitudes por tamaño de la entidad.\\

\begin{tabular}{ m{.26\textwidth} m{.61\textwidth}  }%
	\rowcolor{gray1} {\bf Campo} &  {\bf Descripción} \\ \hline \hline

	id Solicitante & Es el número de folio o el número de solicitud.\\
	\rowcolor{gray1}id Sede & Se refiere al número de folio o de solicitud, antecedido por la clave de la sede.\\
	RFC & Se refiere al RFC de la entidad.\\
	\rowcolor{gray1}Nombre de la Empresa & Es el nombre de la entidad.\\
	Tamaño Empresa & Es de acuerdo a su actividad económica y el número de personas que laboran en la entidad.\\
	\rowcolor{gray1}Entidad Federativa & Es el nombre del estado al que pertenece la entidad.\\
	Total Personas & Es el número total de personas que laboran en la entidad.\\
	\rowcolor{gray1}Núm. Registro & Es el número de registro de la entidad.\\
	Fecha Inicio Registro & Es la fecha en que la entidad llevo a cabo su registro.\\
	\rowcolor{gray1}Fecha Fin Registro & Es la fecha de vencimiento del registro de la entidad.\\
	Nombre Rep. Legal & Es el nombre del representante legal.\\
	\rowcolor{gray1}Correo-E & Es el correo electrónico del representante legal.\\
\end{tabular}

%%---------------------------------------------------------
%\subsection{Solicitudes no revisadas}
%\label{appendix:Reportes:PeopleSoft:SolicitudesNoRevisadas}
%%Solicitudes que no han sido asigandas al coordinador
%%Transcurrido el día hábil, de haber llegado la solicitud, deberá de generarse un reporte  de aquellas que no se hayan revisado.
%El siguiente reporte puede ser consultado por fecha, ID o nombre de la solicitud.
%
%\begin{tabular}{ m{.26\textwidth} m{.61\textwidth}  }%
%	\rowcolor{gray1} {\bf Campo} &  {\bf Descripción} \\ \hline \hline
%
%	Tipo solicitud & Es el tipo de institución o empresa al que pertenece la solicitud, pueden ser: personas físicas con actividad empresarial (PFE), empresas (CEM), instituciones privadas no lucrativas (IPL), instituciones de enseñanza superior - sedes (IES), instituciones de enseñanza superior - subsedes (IES), centros de investigación - sedes (CI), centros de investigación - subsedes(CI), instituciones de la administración pública - sedes (IDAP), instituciones de la administración pública - subsedes (IDAP).\\
%	\rowcolor{gray3}Sede & Se conforma por las siglas del tipo de solicitud más el número de CVU de la persona enlace de la institución o empresa.\\
%	Subsede & Se conforma por las siglas de la subsede más el número de CVU de la persona enlace de la institución o empresa.\\
%	\rowcolor{gray3}Nombre de la solicitud & Es el nombre de la empresa o institución.\\
%	Fecha de envío & Fecha de envío de la solicitud.\\
%	\rowcolor{gray3}Días de retraso & Número de días que han transcurrido desde la fecha de envío hasta el momento, sin que la solicitud haya sido revisada.\\
%\end{tabular}

%%---------------------------------------------------------
%\subsection{Resultado de la evaluación}
%\label{appendix:Reportes:Dictaminador:ResultadoDeLaEvaluacion}
%
%REVISAR ESTA CON GABY\\
%
%\begin{tabular}{ m{.26\textwidth} m{.61\textwidth}  }%
%	\rowcolor{gray1} {\bf Campo} &  {\bf Descripción} \\ \hline \hline
%
%	Tipo solicitud & Es el tipo de institución o empresa al que pertenece la solicitud, pueden ser: personas físicas con actividad empresarial (PFE), empresas (CEM), instituciones privadas no lucrativas (IPL), instituciones de enseñanza superior - sedes (IES), instituciones de enseñanza superior - subsedes (IES), centros de investigación - sedes (CI), centros de investigación - subsedes(CI), instituciones de la administración pública - sedes (IDAP), instituciones de la administración pública - subsedes (IDAP).\\
%	\rowcolor{gray1}Sede & Se conforma por las siglas del tipo de solicitud más el número de CVU de la persona enlace de la institución o empresa.\\
%	Subsede & Se conforma por las siglas de la subsede más el número de CVU de la persona enlace de la institución o empresa.\\
%	\rowcolor{gray1}Nombre de la solicitud & Es el nombre de la empresa o institución.\\
%	Coordinador & Nombre del coordinador que asigna la solicitud.\\
%	\rowcolor{gray1}Dictaminador & Nombre del dictaminador que asigno la solicitud.\\
%	Dictamen & Puede ser: aceptado o mayor información.\\
%\end{tabular}

\newpage
%=========================================================
\section{Archivos Excel generados por la dirección del RENIECYT}
\label{appendix:Reportes:DireccionReniecyt}

La siguiente sección describe los archivos excel generados y administrados por la \refActor{Dirección del RENIECYT}, en los cuales se registran los estados que las solicitudes pueden tener, desde que llegan al módulo de administración y hasta obtener un dictamen. \\

%\noindent En estos archivos se registran los estados en los que se encuentra cada una de las solicitudes en determinada etapa del proceso de registro RENIECYT.


% %---------------------------------------------------------
% \subsection{Revisión de solicitudes}
% \label{appendix:Reportes:DireccionRENIECYT:RevisionDeSolicitudes}
% %Este reporte es generado cuando se han asignado las solicitudes a los coordinadores y enviado el correo electrónico a los usuarios de que su solicitud ha sido enviada a evaluación
% %Este reporte puede sustituir todos los correos que llegan a la dirección de reniecyt, ya que contiene el estatus de las solicitudes a lo largo del proceso.
% %este también debe ser llamado por la reasignación
%
% El siguiente reporte puede ser consultado por fecha, ID o nombre de la institución.
%
% \begin{tabular}{ m{.26\textwidth} m{.61\textwidth}  }%
% 	\rowcolor{gray1} {\bf Campo} &  {\bf Descripción} \\ \hline \hline
%
% 	Tipo solicitud & Es el tipo de institución o empresa al que pertenece la solicitud, pueden ser: personas físicas con actividad empresarial (PFE), empresas (CEM), instituciones privadas no lucrativas (IPL), instituciones de enseñanza superior - sedes (IES), instituciones de enseñanza superior - subsedes (IES), centros de investigación - sedes (CI), centros de investigación - subsedes(CI), instituciones de la administración pública - sedes (IDAP), instituciones de la administración pública - subsedes (IDAP).\\
% 	\rowcolor{gray1}Sede & Se conforma por las siglas del tipo de solicitud más el número de CVU de la persona enlace de la institución o empresa.\\
% 	Subsede & Se conforma por las siglas de la subsede más el número de CVU de la persona enlace de la institución o empresa.\\
% 	\rowcolor{gray1}Nombre de la solicitud & Es el nombre de la empresa o institución.\\
% 	Entidad federativa & Entidad federativa de la empresa o institución.\\
% 	\rowcolor{gray1}Estructura jurídica o tamaño de la empresa & Puede ser si es una empresa micro, pequeña, mediana o grande.\\
% 	Correo del enlace reniecyt & Correo electrónico de la persona que funge como enlace de la empresa o institución.\\
% 	\rowcolor{gray1}Falta de información o anexos incompletos & Se colocan los comentarios cuando la solicitud no puede continuar con el proceso de evaluación por falta de información o anexos incorrectos.\\
% 	Comentarios al dictaminador & Cuando la solicitud es turnada al coordinador, se colocan los comentarios enviados a éste.\\
% 	\rowcolor{gray1}Aceptada para dictamen & Si la solicitud fue aceptada o pendiente de mayor información.\\
% 	Fecha de envío & Fecha de envío de la solicitud.\\
% 	\rowcolor{gray1}Días de retraso & Número de días que han transcurrido desde la fecha de envío hasta el momento, sin que la solicitud haya sido revisada.\\
% \end{tabular}


%---------------------------------------------------------
\subsection{Histórico}
\label{appendix:Reportes:DireccionRENIECYT:Historico}

Este archivo se utiliza por la \refActor{Dirección del RENIECYT} en el proceso de revisión, para determinar si las solicitudes recibidas son \hyperref[glosario:SolicitudNueva]{nuevas} o de \hyperref[glosario:SolicitudDeReinscripcion]{reinscripción}. Al concluir el proceso RENIECYT y sólo si las solicitudes son aceptadas, se registran en este archivo.\\

\begin{tabular}{ m{.26\textwidth} m{.61\textwidth}  }%
	\rowcolor{gray1} {\bf Campo} &  {\bf Descripción} \\ \hline \hline

	\rowcolor{gray1} Núm. & La secuencia \\
	 ID Solicitud & Número que sirve como credencial\\
	\rowcolor{gray1} Clave & Clave de registro \\
	Reinscripción & Se indica si hay reinscripción\\
	\rowcolor{gray1} Fecha de Registro & El día que se obtuvo el registro \\
	Vencimiento de Registro & El día que caduca el registro\\
	\rowcolor{gray1} Reinscripción  &Se indica si hay reinscripción \\
	Fecha de Registro & El día que se solicita la reinscripción \\
	\rowcolor{gray1} Vencimiento de Registro & El día que termina el registro \\
	Último Registro & Número de registro\\
	\rowcolor{gray1}Fecha de Registro & Día en el que se registra \\
	Vencimiento de Registro& Día en el que se  vence el registro\\
	\rowcolor{gray1} Carta de Confidencialidad &  Si firmó o no firmó\\
	Fecha de Firma Electrónica & El día que firmó \\
	\rowcolor{gray1} Convenio de Confidencialidad (Firma Autógrafa) & Firma electrónica antiguo documento\\
	Nombre de Solicitud & Nombre de persona o empresa solicitante\\
	\rowcolor{gray1}Primer No. De Registro& 1er. Número de registró que se obtuvo \\
	Fecha	 & Fecha de registro en la que se realizó\\
	\rowcolor{gray1} Reunión & Sesión en la que se revisaba\\
	No. de Expediente &Número de expediente, esta revisión era manual  \\
	\rowcolor{gray1} No. de Registro  &Año en el  que se registra y un número anexo\\
	Fecha de Registro& Día en el que se realizaba registro \\
	\rowcolor{gray1} No. de Registro   & Año en el  que se registra y un número anexo\\
	Fecha de Registro &  Día en el que se realizaba registro\\
	\rowcolor{gray1}No. de Registro   &Año en el  que se registra y un número anexo \\
	Fecha de Registro&  Día en el que se realizaba registro\\
	\rowcolor{gray1} No. de Registro   & Año en el  que se registra y un número anexo\\
	Fecha de Registro &  Día en el que se realizaba registro\\
	\rowcolor{gray1}No. de Registro   &Año en el  que se registra y un número anexo \\
	Fecha de Registro &  Día en el que se realizaba registro\\
	\rowcolor{gray1} Observación&Observaciones que se hacen para el estatus de las solicitudes \\
\end{tabular}


%---------------------------------------------------------
\subsection{Cambio de Estatus}
\label{appendix:Reportes:DireccionRENIECYT:CambioEstatus}

Es un archivo excel utilizado por la \refActor{Dirección del RENIECYT} en el proceso de revisión, para determinar si las solicitudes son de \hyperref[glosario:SolicitudDeMI]{mayor información}, de \hyperref[glosario:Reconsideracion]{recurso de reconsideración} o un \hyperref[glosario:RecursoDeRevision]{recurso de revisión}.\\

\noindent Existe un archivo de cambio de estatus por cada fecha de sesión de la \refActor{Comisión Interna de Evaluación}, en el cual se registran las solicitudes que serán evaluadas en dicha sesión y el dictamen que obtendrán después de la sesión. Las solicitudes sombreadas de color azul son aquellas solicitudes que obtuvieron un estatus de \hyperref[glosario:SolicitudDeMI]{mayor información} después de la sesión, las de color verde son las solicitudes que resultaron negadas después de la sesión.\\

%, ya sea en su \hyperref[glosario:Reconsideracion]{recurso de reconsideración} o en su \hyperref[glosario:RecursoDeRevision]{recurso de revisión}.
%cambio de estatus se actualiza al obtener un dictamen

\begin{tabular}{ m{.26\textwidth} m{.61\textwidth}  }%
	\rowcolor{gray1} {\bf Campo} &  {\bf Descripción} \\ \hline \hline
	ID Solicit & Número de folio \\
	\rowcolor{gray1} Nombre de Entidad & Nombre de la entidad o nombre de la persona física con actividad empresarial \\
	Comentarios & Comentarios adicionales a cerca de la solicitud \\
\end{tabular}


%---------------------------------------------------------
\subsection{Nuevas Bases}
\label{appendix:Reportes:DireccionRENIECYT:NuevasBases}

Es un archivo excel utilizado por la \refActor{Dirección del RENIECYT} para dar seguimiento a las solicitudes que han sido envíadas a evaluar, a través del estatus actual que presenten.\\



\begin{tabular}{ m{.26\textwidth} m{.61\textwidth}  }%
	\rowcolor{gray1} {\bf Campo} &  {\bf Descripción} \\ \hline \hline

	Fecha recepción & El día que se recibe la solicitud  \\
	\rowcolor{gray1} Fecha de revisión& El día en el que se revisa el contenido de la solicitud \\
	Sede& Tipo de formato más el ID Solicitante\\
	\rowcolor{gray1} Sesión & Se deja en blanco \\
	Nombre Institución/Empresa & Nombre de la entidad o Persona Física\\
	\rowcolor{gray1} Ultimo status solicitud& El estatus actual de la solicitud: envíada a evaluar, en proceso de evaluación, por asignar, etc.\\
	Fecha asignación al coordinador & El día que se envía la solicitud al Coordinador \\
	\rowcolor{gray1} Nombre de Coordinador & Nombre del coordinador responsable de cada área.\\
	Evaluador asignado & Dictaminador responsable de evaluar solicitud\\
	\rowcolor{gray1} Estado& Entidad federativa\\
\end{tabular}

%---------------------------------------------------------
\subsection{Bitácora}
\label{appendix:Reportes:DireccionRENIECYT:BitacoraDiaria}

La bitácora, es un archivo en excel en el que se registran todas las solicitudes que llegan al módulo de administración del RENIECYT. En ella se registran los diferentes estados que puede presentar una solicitud a lo largo del proceso de registro.\\

\noindent Por ser un archivo excel, la bitácora presenta una hoja por cada fecha de sesión de la \refActor{Comisión Interna de Evaluación}. Las solicitudes que llegan al módulo de administración se registran en la hoja correspondiente a la fecha sesión que esté por ocurrir. Si una solicitud es regresada por falta de información, está mantendrá su registro en la sesión donde fue registrada, y después de la actualización de información por parte del solicitante, se registrará nuevamente en la sesión que esté por ocurrir, pudiendo ser la misma sesión o una posterior. Si el registro no ocurre en la misma sesión, entonces el registro anterior se ignora y el registro comienza a partir de la sesión que esté por ocurrir. Esto sucede, por que no existe un tiempo límite establecido para corregir la información de la solicitud por parte del solicitante.\\

\begin{tabular}{ m{.26\textwidth} m{.61\textwidth}  }%
	\rowcolor{gray1} {\bf Campo} &  {\bf Descripción} \\ \hline \hline

	Fecha de recepción &El día que se recibe la solicitud\\
	\rowcolor{gray1} Fecha de revisión &El día que se revisa\\
	Hora& La hora en la que se empieza la revisión\\
	\rowcolor{gray1} Tipo de formato &Formato específico del usuario \\
	Tipo de Solicitud&Nueva, Reinscripción, De mayor información, Recursos de consideración \\
	\rowcolor{gray1} ID solicitante& Número de folio \\
	Nombre de la Solicitud & Nombre de entidad o persona física\\
	\rowcolor{gray1} Entidad Federativa& Entidad federativa de acuerdo a su domicilio fiscal\\
	Email enlace reniecyt& Correo electrónico del enlace \\
	\rowcolor{gray1}Tipo estructura jurídica& Puede ser: Micro, Pequeñas, Medianas, Grandes, Institución Públicas, Instituciones privadas, A.C., S.C., Centros de Investigación, Otros \\
	Programa al que quiere aplicar (información adicional)&Información que capturó el usuario en la sección de información adicional  \\
	\rowcolor{gray1} Estatus 1a. vez& Estatus de la solicitud: 1) Se envía a  evaluar, 2) Si se regresa, se anotan los motivos \\
	Fecha de recepción (2a.vez) & Se anota día que se recibe \\
	\rowcolor{gray1} Fecha de revisión y hora& Se anota día que se recibe y hora en el que se revisa \\
	Estatus 2a. vez& Estatus de la solicitud: 1) Se envía a  evaluar, 2) Si se regresa, se anotan los motivos\\
	\rowcolor{gray1} Fecha de recepción(3a. vez) & Se anota día que se recibe y el día y hora en el que se revisa\\
	Estatus 3a. vez& Se envía a evaluar\\
	\rowcolor{gray1} Área evaluadora& Área que evaluará la solicitud \\
	Evaluar en sesión de fecha &Sesión en la que se manda a evaluar solicitudes con estatus de mayor información \\
	\rowcolor{gray1} Hora&Hora en la que se mandó a evaluar \\
\end{tabular}


% %=========================================================
% \section{Reportes generados por el Coordinador}
% \label{appendix:Reportes:Coordinador}
%
% %---------------------------------------------------------
% \subsection{Solicitudes remitidas a dictamen}
% \label{appendix:Reportes:Dictaminador:SolicitudesRemitidasADictamen}
%
% Este reporte puede ser consultado por la \refActor{Dirección de RENIECYT} y el \refActor{Coordinador}. La \refActor{Dirección de RENIECYT} puede consultar el reporte por coordinación, solicitud o fecha de envío a dictamen, mientras que el \refActor{Coordinador} puede hacerlo por solicitud, fecha de envío o dictaminador.
%
% \begin{tabular}{ m{.26\textwidth} m{.61\textwidth}  }%
% 	\rowcolor{gray1} {\bf Campo} &  {\bf Descripción} \\ \hline \hline
%
% 	Tipo solicitud & Es el tipo de institución o empresa al que pertenece la solicitud, pueden ser: personas físicas con actividad empresarial (PFE), empresas (CEM), instituciones privadas no lucrativas (IPL), instituciones de enseñanza superior - sedes (IES), instituciones de enseñanza superior - subsedes (IES), centros de investigación - sedes (CI), centros de investigación - subsedes(CI), instituciones de la administración pública - sedes (IDAP), instituciones de la administración pública - subsedes (IDAP).\\
% 	\rowcolor{gray1}Sede & Se conforma por las siglas del tipo de solicitud más el número de CVU de la persona enlace de la institución o empresa.\\
% 	Subsede & Se conforma por las siglas de la subsede más el número de CVU de la persona enlace de la institución o empresa.\\
% 	\rowcolor{gray1}Nombre de la solicitud & Es el nombre de la empresa o institución.\\
% 	Fecha de envío a dictamen & Fecha en que se envió la solicitud a dictamen.\\
% 	\rowcolor{gray1}Dictaminador & Nombre del dictaminador que asigno la solicitud.\\
% 	3 días hábiles & Se coloca la fecha en que debe ser terminada la evaluación para identificar la fecha de entrega.\\
% 	\rowcolor{gray1}Solicitudes con más de 3 días hábiles en dictamen & Estatus para determinar si una solicitud ha salido del rango de tiempo establecido para su evaluación. \\
% 	Dictamen & Puede ser: aceptado o mayor información.\\
% \end{tabular}


% %---------------------------------------------------------
% \subsection{Solicitudes NO remitidas a dictamen}
% \label{appendix:Reportes:Dictaminador:SolicitudesNORemitidasADictamen}
% %En caso de que el coordinador, pasado 1 día hábil de haber recibido la solicitud para enviar a dictamen, no la haya remitido al dictaminador se deberá generar un reporte al reniecyt
%
%  con la siguiente información
%
% \begin{tabular}{ m{.26\textwidth} m{.61\textwidth}  }%
% 	\rowcolor{gray1} {\bf Campo} &  {\bf Descripción} \\ \hline \hline
%
% 	Tipo solicitud & Es el tipo de institución o empresa al que pertenece la solicitud, pueden ser: personas físicas con actividad empresarial (PFE), empresas (CEM), instituciones privadas no lucrativas (IPL), instituciones de enseñanza superior - sedes (IES), instituciones de enseñanza superior - subsedes (IES), centros de investigación - sedes (CI), centros de investigación - subsedes(CI), instituciones de la administración pública - sedes (IDAP), instituciones de la administración pública - subsedes (IDAP).\\
% 	\rowcolor{gray1}Sede & Se conforma por las siglas del tipo de solicitud más el número de CVU de la persona enlace de la institución o empresa.\\
% 	Subsede & Se conforma por las siglas de la subsede más el número de CVU de la persona enlace de la institución o empresa.\\
% 	\rowcolor{gray1}Nombre de la solicitud & Es el nombre de la empresa o institución.\\
% 	Fecha de envío a dictamen & Fecha en que se envió la solicitud a dictamen.\\
% 	\rowcolor{gray1}Coordinador & Nombre del coordinador que asigna la solicitud.\\
% \end{tabular}
%
%
% %=========================================================
% \section{Reportes generados por el dictaminador}
% \label{appendix:Reportes:Dictaminador}


%
% %---------------------------------------------------------
% \subsection{Solicitudes no dictaminadas}
% \label{appendix:Reportes:Dictaminador:SolicitudesNoDictaminadas}
%
% \begin{tabular}{ m{.26\textwidth} m{.61\textwidth}  }%
% 	\rowcolor{gray1} {\bf Campo} &  {\bf Descripción} \\ \hline \hline
%
% 	Tipo solicitud & Es el tipo de institución o empresa al que pertenece la solicitud, pueden ser: personas físicas con actividad empresarial (PFE), empresas (CEM), instituciones privadas no lucrativas (IPL), instituciones de enseñanza superior - sedes (IES), instituciones de enseñanza superior - subsedes (IES), centros de investigación - sedes (CI), centros de investigación - subsedes(CI), instituciones de la administración pública - sedes (IDAP), instituciones de la administración pública - subsedes (IDAP).\\
% 	\rowcolor{gray1}Sede & Se conforma por las siglas del tipo de solicitud más el número de CVU de la persona enlace de la institución o empresa.\\
% 	Subsede & Se conforma por las siglas de la subsede más el número de CVU de la persona enlace de la institución o empresa.\\
% 	\rowcolor{gray1}Nombre de la solicitud & Es el nombre de la empresa o institución.\\
% 	Coordinador & Nombre del coordinador que asigna la solicitud.\\
% 	\rowcolor{gray1}Dictaminador & Nombre del dictaminador que asigno la solicitud.\\
% \end{tabular}

