%- - - - - - - - - - - - - - - - - - - - - - - - 

\begin{AreaOportunidad}{PGPA-AO1}
	\item[Área:] {Unidades Académicas\footnote{Ver \refElem{UnidadAcademica} }, \refElem{DES} y \refElem{DAE} }.
	\item[Procesos:] Gestión de Programas Académicos.
	\item[Problema:] La comunicación entre las Unidades Académicas y la \refElem{DES} para compartir la información del registro de nuevos programas académicos o el rediseño de los mismos, se realiza actualmente sin una interacción con el \refElem{SAES}, que es donde al final se registran los planes de estudio oficiales dentro del instituto. Lo anterior deriva en un retardo en el registro de la información oficial y en la detección de errores en la misma.
	\item[Área de oportunidad:] Mejorar la comunicación a través de una interfaz para la gestión de planes de estudio por parte de las Unidades Académicas, donde la \refElem{DES} pueda observar la información y dar su visto bueno, una vez realizada su validación de los planes de estudio por medio del proceso vigente.
	\item[Impacto:] Medio.
	\item[Alcance:] Dentro del ámbito del CALMÉCAC recae considerar que la Unidad Académica pueda gestionar sus planes de estudio o el rediseño de los mismos, el cual pueda ser observado por la \refElem{DES} y avalarlo una vez que ha realizado su proceso de validación. La \refElem{DAE} deberá mantener las facultades de realizar los cambios que sean necesarios para actualizar la información y validar el registro o los cambios realizados.\\
	
\end{AreaOportunidad}

%==================================


