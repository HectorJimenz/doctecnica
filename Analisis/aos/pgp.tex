%- - - - - - - - - - - - - - - - - - - - - - - - 
\begin{AreaOportunidad}{PGP-AO1}
	\item[Área:] \refElem{DCH}.
	\item[Procesos:] Gestión de Profesores.
	\item[Problema:] Actualmente existen dos sistemas oficiales (SAES y SIEE) que manejan la información de la plantilla docente, dicha información no coincide debido a que se manejan fuentes de información diferentes.
	La información de la situación laboral de los docentes es dinámica, la cual es gestionada por la \refElem{DCH}, el SAES no cuenta con un mecanismo que permita tener dicha información vigente.\\
	Cuando se generan cambios en la estructura académica una vez iniciado el semestre, no existe un mecanismo de notificación y actualización de la información para \refElem{DAE}, \refElem{DES} y \refElem{DEMS}.
	
	%Además el concepto de Profesor Invitado no se contempla en el sistema SAES.
	%\item[Área de oportunidad:] 
	\item[Impacto:] Medio.
	\item[Alcance:] Se propone obtener la información de los profesores directamente de la \refElem{DCH}. %Una vez que la estructura académica ha sido aprobada por Capital Humano, la \refElem{DES}  y la \refElem{DEMS}, se generará un mecanismo que informe de los cambios realizados a éstas por parte de la unidad académica, con la finalidad de enterar a la DES y Capital Humano
	
\end{AreaOportunidad}


%=============================

