%- - - - - - - - - - - - - - - - - - - - - - - -

\begin{AreaOportunidad}{PRE-AO1}
	\item[Área:] \refElem{UnidadAcademica} y \refElem{DES}.
	\item[Procesos:] Registro de Evaluaciones.
	\item[Problema:] Actualmente los procesos que se tienen para el registro de calificaciones presentan vulnerabilidades. Existen huecos de seguridad a lo largo de todo el proceso.
	\item[Área de Oportunidad:] Fortalecer dichos procesos implementando la firma digital como mecanismo de seguridad para verificar la autenticidad y garantizar el no repudio de cualquier calificación, así como un folio digital para el subproceso de modificación de calificaciones como un identificador único en el Instituto Politécnico Nacional para el acta que se generará.
	\item[Impacto:] Alto.
	\item[Alcance:] Dentro del CALMÉCAC se integrará un mecanismo de seguridad (firma electrónica) que aumente los servicios de integridad y permita no repudio.
\end{AreaOportunidad}

%- - - - - - - - - - - - - - - - - - - - - - - -

\begin{AreaOportunidad}{PRE-AO2}
	\item[Área:] \refElem{UnidadAcademica}.
	\item[Procesos:] Registro de Evaluaciones.
	\item[Problema:] El mantener un Kárdex físico presenta un problema cuando la población estudiantil es demasiado grande, así como de salud después de varios años.
	\item[Área de Oportunidad:]El Kárdex se puede implementará de manera digital la actualización podría ser automática en cuanto el profesor subiera la calificación y la cantidad de alumnos no presentaría ningún problema.
	\item[Impacto:] Medio.
	\item[Alcance:] Con la implementación de diversos mecanismos de seguridad que tendrá el CALMÉCAC se puede almacenar de manera digital  el kárdex para cada unidad académica.
\end{AreaOportunidad}

%- - - - - - - - - - - - - - - - - - - - - - - -

\begin{AreaOportunidad}{PRE-AO3}
	\item[Área:] \refElem{UnidadAcademica}.
	\item[Procesos:] Registro de Evaluaciones.
	\item[Problema:] El sistema no diferencia entre las diversas modalidades de estudios.
	\item[Área de Oportunidad:]Diferenciar las diversas modalidades escolares cuando se realicen las evaluaciones ordinarias.
	\item[Impacto:] Alto.
	\item[Alcance:] El CALMÉCAC podrá distinguir entre las diferentes modalidades de estudios.
\end{AreaOportunidad}

%- - - - - - - - - - - - - - - - - - - - - - -

\begin{AreaOportunidad}{PRE-AO4}
	\item[Área:] \refElem{UnidadAcademica} y \refElem{DES}.
	\item[Procesos:] Registro de Evaluaciones.
	\item[Problema:] El proceso de inscripción a los Exámenes a Título de Suficiencia se hace de manera presencial por lo que requiere de bastante tiempo y personal para poder cubrir la demanda de estos.
	\item[Área de Oportunidad:] El registro de los Exámenes a Título de Suficiencia puede optimizarse. Si se hiciera de manera digital por medio de un sistema que indicará qué alumnos son candidatos a realizar dicho examen.
	\item[Impacto:] Medio.
	\item[Alcance:] El CALMÉCAC facilitará el registro de los alumnos a los Exámenes a Título de Suficiencia gestionando qué alumnos son candidatos a realizar dicho examen.
\end{AreaOportunidad}

%- - - - - - - - - - - - - - - - - - - - - - - -

\begin{AreaOportunidad}{PRE-AO5}
	\item[Área:] \refElem{UnidadAcademica} .
	\item[Procesos:] Registro de Evaluaciones.
	\item[Problema:] Falta validación de inicio de evaluaciones y opción a corregir las fechas de inicio y fin de éstas.
	\item[Área de Oportunidad:] El inicio del proceso de registrar las evaluaciones podría ir ligado al calendario escolar de la modalidad escolar a la que pertenece.  
	\item[Impacto:] Medio.
	\item[Alcance:] El CALMÉCAC facilitara el manejo de periodos de evaluación.
\end{AreaOportunidad}


