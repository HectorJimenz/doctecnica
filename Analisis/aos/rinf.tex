%- - - - - - - - - - - - - - - - - - - - - - - - 
\begin{AreaOportunidad}{PGINF-AO1}
	\item[Área:] \refElem{DES}.
	\item[Procesos:] Proceso de Registro de Infraestructura.
	\item[Problema:] 	No se cuenta con la información precisa de donde se imparten las clases debido a que  no se lleva un control adecuado de los espacios, ya sea internos o externos, que están habilitados para impartir clases de los diversos programas académicos que ofrece el Instituto.
	\item[Área de oportunidad:] Implementar un proceso que permita la construcción de un catálogo institucional a partir de la información que cada unidad académica proporcione, en donde se tengan registrados todos los espacios habilitados para impartir clases.  Además, permitiría identificar los espacios que no pertenecer al Instituto, como lo son clínicas u hospitales.  
	
	\item[Impacto:] bajo.
	\item[Alcance:] Implementar un proceso que permita la construcción de un catálogo institucional de los espacios usados para impartir las unidades de aprendizaje, a partir de la información que cada unidad académica proporcione. Un catálogo con estás características permite identificar como se usan desde la perspectiva académica los espacio que pertenecen al Instituto, adicionalmente se podrá identificar aquellos que no le pertenecen  como son clínicas y hospitales.

\end{AreaOportunidad}






%\begin{AreaOportunidad}{PI-AO1}
%	\item[Área:] \refElem{DAE}
%	\item[Procesos:] Admisión e Ingreso.
%	\item[Área de Oportunidad:] El mecanismo actual en el que el Departamento de Informática comunica al Departamento de Supervisión los aspirantes aceptados y su asignación a las Escuelas ({\em la cual se realiza mediante vistas de Oracle que deben ser consultadas manualmente y mediante ``scripts programados''.}) provoca que el trabajo para mantener al SAES actualizado implique mucho trabajo y tiempo.
%	\item[Mejoras:] Mejorar dicha comunicación mediante la integración de ambos sistemas ({\em el de informática para el proceso de admisión y el Calmécac}) mediante Servicios Web, Formatos especializados que permitan automatizar el intercambio de información o la integración de ambas Bases de datos.
%	\item[Impacto:] Medio.
%	\item[Alcance:] Dentro del ámbito del proyecto Calmécac recae considerar la mejora de la comunicación entre estas áreas, pero no, adecuaciones o mejoras al Sistema de Admisión, las cuales de ser necesarias requerirán del apoyo del departamento de Informática para lograrlas.
%\end{AreaOportunidad}


