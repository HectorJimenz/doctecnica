%- - - - - - - - - - - - - - - - - - - - - - - - 
\begin{AreaOportunidad}{PEA-AO1}
	\item[Área:] \refElem{UnidadAcademica}.
	\item[Procesos:] Solicitud de horas de interinato.
	\item[Problema:] Las escuelas solicitan horas de interinato sin agotar las capacidades de la planta docente.
	\item[Área de Oportunidad:] Generar un mecanismo que ayude a identificar a los profesores que no cubren su carga máxima.
		\item[Impacto:] Medio
	\item[Alcance:] Implementación de una estrategia que facilite identificar a los profesores que no cubren su carga maxima.
	\end{AreaOportunidad}


%- - - - - - - - - - - - - - - - - - - - - - - -

\begin{AreaOportunidad}{PEA-AO2}
	\item[Área:] \refElem{UnidadAcademica},\refElem{DAE},\refElem{DCH}
	\item[Procesos:] Registro de estructura académica, Registro de horarios en el SAES, Registro en el SRN
	\item[Problema:] Toda la información que contiene la estructura se captura más de una ocasión manualmente y se hace para cada sistema (SIIEE, SRN y SAES), por lo que es no es complicado contar con una versión única de la información de los profesores, situación que empeora cuando en alguno de os sistemas se realiza un cambio, éste no se refleja en los demás sistemas.	
		\item[Área de Oportunidad:]Desarrollar un mecanismo que permita sincronizar los diferentes sistemas que se utilizan en el Instituto que están relacionados con la gestión de profesores y su asignación de unidades académicas.
	\item[Impacto:] Medio
	\item[Alcance:] Generación de estrategias de comunicación, como pueden ser los servicios web.
\end{AreaOportunidad}

%- - - - - - - - - - - - - - - - - - - - - - - - -
%
\begin{AreaOportunidad}{PEA-AO3}
	\item[Área:] \refElem{UnidadAcademica}.
	\item[Procesos:] Identificación de inconsistencias
	\item[Problema:] No contar con el soporte documental cuando la carga máxima de un profesor esta por arriba o por debajo de la carga máxima.
	\item[Área de Oportunidad:]Generar un repositorio que facilite el acceso al soporte documental de los casos de los profesores tienen una carga diferente a la m-axima.
	\item[Impacto:] Medio
	\item[Alcance:] Generar un mecanismo de consulta que permita conocer porque un profesor no cubre su carga máxima.
\end{AreaOportunidad}
%
%%- - - - - - - - - - - - - - - - - - - - - - - - -
%
%
%\begin{AreaOportunidad}{PEA-AO4}
%	\item[Área:] \refElem{UnidadAcademica},\refElem{DES}.
%	\item[Procesos:] Generación de propuesta de estructura académica.
%	\item[Problema:] Que un profesor tenga adeudo de horas, es decir, que no esté en carga máxima y no justifique el por qué.
%	\item[Impacto:] Que no se permita si no tiene un soporte documental ya que se puede prestar a favoritismos.
%	\item[Alcance:] Agregar las validaciones necesarias para la asignación de unidades de aprendizaje a profesores.
%\end{AreaOportunidad}


%- - - - - - - - - - - - - - - - - - - - - -  - - - -  - -

\begin{AreaOportunidad}{PEA-AO4}
	\item[Área:] \refElem{}.
	\item[Procesos:] Evaluación de la estructura académica, Envío de propuesta de reestructuración académica
	\item[Problema:] Gestionar la aprobación de la estructura académica es complejo por la cantidad de variable que están inmersas en el proceso, el mecanismo actual de comunicación es deficiente.
	\item[Área de Oportunidad:]Construir una estrategia que ayude a facilitar la gestión de la estructura Académica.
	\item[Impacto:] Alto
	\item[Alcance:] Construir un módulo que ayude a la gestión de la estructura académica.
\end{AreaOportunidad}


%- - - - - - - - - - - - - - - - - - - - - - - - - - 

