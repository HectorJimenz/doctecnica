%- - - - - - - - - - - - - - - - - - - - - - - - 
\begin{AreaOportunidad}{PR-AO1}
	\item[Área:] \refElem{DAE}.
	\item[Procesos:] Reinscripción.
	\item[Problema:] El SAES actualmente no distingue entre modalidad mixta y presencial.
	\item[Impacto:] Bajo.
	\item[Alcance:] Se propone que el Calmécac distinga entre las dos modalidades.
\end{AreaOportunidad}

%- - - - - - - - - - - - - - - - - - - - - - - -

\begin{AreaOportunidad}{PR-AO2}
	\item[Área:] \refElem{DAE}.
	\item[Procesos:] Reinscripción.
	\item[Problema:] El SAES fue diseñado sin considerar que un programa académico puede tener varias especialidades.
	\item[Impacto:] Medio.
	\item[Alcance:]Se propone que el Calmécac considere que un programa académico puede tener varias especialidades y que valide que un alumno únicamente pueda inscribir unidades de aprendizaje de la especialidad registrada.
\end{AreaOportunidad}

%- - - - - - - - - - - - - - - - - - - - - - - - - 


\begin{AreaOportunidad}{PR-AO3}
	\item[Área:] \refElem{DAE}.
	\item[Procesos:] Reinscripción.
	\item[Problema:] El módulo de dictámenes que considera el SAES no satisface las necesidades de las reinscripciones.
	\item[Impacto:] Alto.
	\item[Alcance:]Construir un módulo para la gestión de dictámenes que ayude a conocer la situación escolar del alumno, a fin de determinar las restricciones a aplicar para su reinscripción con base en el dictamen.	
\end{AreaOportunidad}

%- - - - - - - - - - - - - - - - - - - - - - - - - -

\begin{AreaOportunidad}{PR-AO4}
	\item[Área:] \refElem{DAE}.
	\item[Procesos:] Reinscripción.
	\item[Problema:] El SAES no permite identificar cuando un alumno no tiene una materia desfasada en su programa académico.
	\item[Impacto:] Alto.
	\item[Alcance:]Se propone que el Calmécac Identifique las materias desfasadas y limite la reinscripción de acuerdo al reglamento general de estudios.	
\end{AreaOportunidad}

%- - - - - - - - - - - - - - - - - - - - - - - - - - -

\begin{AreaOportunidad}{PR-AO5}
	\item[Área:] \refElem{DAE}.
	\item[Procesos:] Reinscripción.
	\item[Problema:] Es complicado acreditar en el SAES una materia por saberes previamente adquiridos, debido a que cuando un alumno acredita una unidad de aprendizaje por este medio el SAES, se tiene que realizar la baja de forma manual.
	\item[Impacto:] Bajo.
	\item[Alcance:]Se propone que el Calmécac permita la gestión de registro de calificaciones de unidades de aprendizaje previamente adquiridos.
\end{AreaOportunidad}

%- - - - - - - - - - - - - - - - - - - - - - - - - - -

\begin{AreaOportunidad}{PR-AO6}
	\item[Área de Oportunidad:] PR-AO6
	\item[Área:] \refElem{DAE}.
	\item[Procesos:] Reinscripción.
	\item[Problema:] Plataforma en Modalidad Virtual no tiene vinculación con SAES.
	\item[Impacto:] Alto.
	\item[Alcance:] Buscar un mecanismo de comunicación entre la modalidad virtual y la presencial.
\end{AreaOportunidad}

%- - - - - - - - - - - - - - - - - - - - - - - - - - - -

\begin{AreaOportunidad}{PR-AO7}
	\item[Área:] \refElem{DAE}.
	\item[Procesos:] Reinscripción.
	\item[Problema:] El SAES está desarrollado para modalidad escolarizada, en cuestión de los horarios, no muestra horarios para los programas académicos en modalidad mixta.
	\item[Impacto:] Alto.
	\item[Alcance:] Distinguir entre la modalidad mixta y presencial, así como en que modalidad el estudiante aprueba sus materias.
\end{AreaOportunidad}

%- - - - - - - - - - - - - - - - - - - - - - - - - - - - 
