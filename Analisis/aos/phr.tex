
\begin{AreaOportunidad}{PRH-AO1}
	\item[Área:] \refElem{DES}, \refElem{JefeDeDepartamento}, \refElem{SubdireccionAcademica}.
	\item[Procesos:] Registro de Horarios.
	
	\item[Problema:] La generación de horarios por parte de las unidades académicas se realiza en sistemas independientes al SAES, posteriormente se capturan en este, sin embargo el SAES no brinda apoyo en identificar traslapes entre salones, profesores, que una unidad de aprendizaje que pertenece a un grupo no se traslape en el mismo horario con otra unidad de aprendizaje del mismo grupo. 
	
	\item[Área de Oportunidad:]Desarrollar un Subsistema que pueda enriquecer la estructura académica. Con esto se desacoplaría la parte de profesores y carga académica con fines presupuestales y laborales, permitiendo tener disponible la información que requiere el \refElem{Calmecac} para la Gestión Escolar.
		\begin{itemize}
			\item Identificar los profesores con traslapes (entre, carreras)
			\item Las aulas que se traslapan.
			\item Identificar los profesores que no tienen su carga máxima e acuerdo a su categoría.
			\item Conocer el catálogo de materias impartidas por un profesor.
			\item Contar en el sistema con el soporte documental que justifica la carga (cumplida o no) de un profesor (cuando no es frente a grupo).
			\item Que el sistema sugiera que materias puede impartir un profesor si no cubre su carga máxima considerando:
				\begin{enumerate}
					\item Materias impartidas.
					\item Horario del profesor.
					\item Materias disponibles.
				\end{enumerate}
			\item Que el sistema considere que hay más de una clave presupuestal y tipo de contrato por profesor.
			\item El Calmécac proveerá parte de la funcionalidad del SIIEE y una interfaz de comunicación con el SIEE para evitar la doble captura, así el SIEE podrá consultar o adquirir la última versión de la Estr. Acad. cuando lo desee.
		\end{itemize}
	
	
	%No existe el concepto de carga de cada profesor, mínima ni máxima. estructura academica
	%- El SAES no permite verificar, controlar, identificar, reportar, profesores que no están cubriendo la carga que les corresponde ni la causal de dicha situación que permita al personal de la Unidad  o CH actuar de manera correcta.
	
	
	
	%\item[Área de oportunidad:]
	%	Generación asistida en la generación de  plantillas grupales que incluyan unidades de Aprendizaje, horarios preestablecidos indicando: días, horas y espacios para la impartición de las diferentes unidades de aprendizaje, permitiendo la selección y asignación del profesor a cubrir esa unidad de aprendizaje.\\
	%Consulta automática de información actualizada de los profesores a través de un servicio web.\\
	%\\
	
	%Evitar traslapes en asignación de horarios por profesor.\\
	%\\
	%.\\
	
	\item[Impacto:] Alta.
	
	\item[Alcance:] Dentro del ámbito del proyecto CALMÉCAC recae considerar las siguientes acciones:
	\begin{itemize}
		\item Copiar una estructura de un semestre similar o equivalente  para usarlos como propuesta o plantilla de horarios.
		\item Evitar traslapes en asignación de horarios por grupo.
		\item Evitar traslapes en asignación de espacios e infraestructura académica.
		\item Ofrecer un mecanismo que permita modificar un horario establecido de acuerdo a las necesidades de la unidad académica.
	\end{itemize}
	
\end{AreaOportunidad}
%=======================

